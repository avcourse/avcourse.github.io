\documentclass[hyperref={unicode=true}]{beamer}
\usepackage[utf8]{inputenx}
\usepackage[russian]{babel}

\usepackage{multicol}

\usepackage[pgf]{dot2texi}
\usepackage{tikz}
\usetikzlibrary{shapes, arrows}

\usepackage{listings}
\usepackage{graphicx}

\usepackage{comment}

\title{Лекция 7. WhyML, автоматизация построения условий верификации}
\author{}
\date{}

\usetheme{Warsaw}

\AtBeginSection[] {
	\begin{frame}{Содержание}
		\tableofcontents[currentsection]
	\end{frame}
}
%\overfullrule=5pt

\begin{document}
	\begin{frame}{}
		\titlepage
	\end{frame}

    \begin{frame}{Цель лекции}
    Познакомиться с языком WhyML и
    научиться использовать его для
    автоматизации дедуктивной верификации.
    \end{frame}

    \section{Основные конструкции WhyML}

    \begin{frame}{Общая характеристика}
    \begin{itemize}
    \item
    Инфраструктура Why3 умеет генерировать
    условия верификации (\texttt{why3 ide file}).
    \item
    \textsl{WhyML} -- функциональный язык
    со средствами формальной спецификации и
    дедуктивной верификации.
    \item
    Обычно выполняет роль промежуточного языка
    между языком программирования и средствами
    верификации.
    \end{itemize}
    \end{frame}

    \begin{frame}{Теории и модули}
    \begin{itemize}
    \item
    Теории -- набор определений и явных целей доказательства
    \item
    Модули -- набор определений и моделей программ, для моделей программ надо
    доказать полную корректность, т.е. сгенерировать условия верификации
    \item
    Синтаксис теории: \texttt{theory Name ... end}
    \item
    Синтаксис модуля: \texttt{module Name ... end}
    \item
    Файл \texttt{*.why} не может содержать модули,
    с модулями файл должен называться \texttt{*.mlw}
    \end{itemize}
    \end{frame}

    \begin{frame}{Типы данных}
    \begin{itemize}
    \item
    Целые числа: \texttt{use import int.Int}
    \item
    Пустой тип \texttt{unit}
    \item
    Декартово произведение \texttt{type T = (T1,T2)}
    \item
    Структуры, поля которых можно делать изменяемыми --
    единственный способ реализовать переменные и операцию
    присваивания \texttt{type S = \{ mutable v: T \}}
    \item
    Поле структуры -- это еще и функция, поэтому его имя
    входит в глобальную область видимости
    \item
    Типы могут быть полиморфными \texttt{type array 't}
    \end{itemize}
    \end{frame}

    \begin{frame}{Операции и выражения}
    \begin{itemize}
    \item
    Любое вычисление -- это выражение
    \item
    \texttt{e1 ; e2 } -- последовательность вычислений выражений
    \item
    \texttt{s.v <- e } -- присваивание в изменяемое поле
    \texttt{v} структуры \texttt{s}
    \item
    \texttt{()} -- пустое вычисление (выражение типа \texttt{unit})
    \item
    \texttt{(expr,expr)} -- кортеж
    \item
    \texttt{\{v = 3\}} -- новое значение типа структуры, у которого
    есть поле \texttt{v} (такой тип единственный)
    \item
    условие \texttt{if-then-else}, циклы \texttt{while, for}
    \end{itemize}
    \end{frame}

    \begin{frame}{Функции и локальные переменные}
    \begin{itemize}
    \item
    Локальная переменная: \texttt{let var = expr in ... }
    \item
    Разбиение кортежа: \texttt{let (v1,v2) = expr in ... }
    \item
    Неанонимная функция -- модель программы:
    \texttt{let function (arguments) : returntype = expr}
    \item
    Можно указать только заголовок функции, тогда вместо \texttt{let}
    надо написать \texttt{val}. Заголовок заканчивается перед
    символом равно.
    \item
    Функция нужна только для доказательства? После \texttt{let}
    надо написать \texttt{ghost}.
    \item
    Это лемма-функция? После \texttt{let}
    надо написать \texttt{lemma}.
    \item
    Нужна рекурсия? Надо приписать после \texttt{let}
    слово \texttt{rec}.
    \end{itemize}
    \end{frame}

    \begin{frame}[fragile]{Пример модуля с функцией}
    \begin{lstlisting}
module Factorial
use import int.Int
let rec fact_rec (n: int): int =
    if n = 0 then 1 else n * fact_rec (n - 1)
type intVar = { mutable v: int; }
let fact_loop (n: int): int =
    let i = { v = 1; } in
    let r = { v = 1; } in
        while i.v <= n do
            r.v <- r.v * i.v;
            i.v <- i.v + 1
        done;
        r.v
end
    \end{lstlisting}
    \end{frame}


    \begin{frame}{Исключения}
    \begin{itemize}
    \item
    Исключения прерывают вычисление
    \item
    Исключения используются для возврата из функции до ее завершения
    \item
    Надо описать тип исключения (конструкция exception)
    \item
    try-блок, with-обработчик, raise-генерация исключения
    \end{itemize}
    \end{frame}

    \begin{frame}{Запись спецификации}
    \begin{itemize}
    \item
    Предусловие и постусловие пишется
    между заголовком функции и символом равно.
    \item
    Предусловие состоит из 1 или более
    конструкций \texttt{requires \{...\} }
    \item
    Постусловие состоит из 1 или более
    конструкций \texttt{ensures \{...\} }
    \item
    Возвращаемое значение записывается как \texttt{result}.
    \item
    Несколько requires означает их конъюнкцию. Аналогично с ensures.
    \item
    Ни одного requires означает истину. Аналогично с ensures.
    \end{itemize}
    \end{frame}

    \begin{frame}[fragile]{Пример спецификаций}
    \begin{lstlisting}
module SqrtExample
use import int.Int
val sqrt (n: int) : int
    requires { n >= 0 }
    ensures { result * result <= n <
    (result + 1) * (result + 1) }
let foo (n: int) : int
    requires { n >= 0 }
    ensures { result >= 0 }
= sqrt n
end
    \end{lstlisting}
    \end{frame}

    \begin{frame}{Методы Флойда}
    \begin{itemize}
    \item
    Точки сечения выбраны жестко: перед проверкой
    условий циклов
    \item
    Индуктивное утверждение записывается так: \texttt{invariant \{...\}}
    и пишется после слова \texttt{do}, если речь о цикле \texttt{while}
    \item
    Несколько \texttt{invariant} возможно, это означает их конъюнкцию
    \item
    Фундированно множество выбрано жестко: целые числа и сравнение меньше
    \item
    Оценочная функция цикла записывается так: \texttt{variant \{...\}}
    и пишется там же, где \texttt{invariant}.
    \item
    Оценочная функция рекурсии записывается так же, но после предусловия.
    \end{itemize}
    \end{frame}

    \begin{frame}[fragile]{Пример спецификации цикла}
    \begin{lstlisting}
module MultLoop
use import int.Int
type intVar = { mutable v: int; }
let loop_test (a b: int): int
    requires { a >= 0 }
    ensures { result = a * b }
= let i = { v = 1; } in
  let r = { v = 0; } in
    while i.v <= a do
        invariant { i.v <= a + 1 }
        invariant { r.v = (i.v - 1) * b }
        variant { a + 1 - i.v }
        r.v <- r.v + b; i.v <- i.v + 1
    done; r.v  end
    \end{lstlisting}
    \end{frame}

    \section{Дополнительные конструкции}

    \begin{frame}{Доказательство примера}
    \begin{itemize}
    \item
    Перепишите блок-схему вычисления квадратного корня в виде функции
    на языке \textsl{WhyML}. Напишите спецификацию и всё необходимое
    для методов Флойда.
    \item
    Завершите доказательство полной корректности: укажите недостающие
    леммы, докажите леммы по индукции или через последовательность
    других лемм, используйте вместо лемм вызов ghost-функции.
    \end{itemize}
    \end{frame}

    \begin{frame}{assert}
    \begin{itemize}
    \item
    Вместо написания отдельной ghost-функции с пустым телом для
    доказательства некоторого утверждения удобнее воспользоваться
    конструкцией assert и не писать эту функцию.
    \item
    Синтаксис такой: \texttt{assert \{ condition \}}
    \item
    Это аналогично ghost-функции с пустым телом и постусловием
    \texttt{condition}. Предусловие собрано из всех условий, которые
    находятся перед assert.
    \end{itemize}
    \end{frame}

    \begin{frame}[fragile]{Пример с assert}
    \begin{lstlisting}
module Theorems

use import int.Int

let theorem (x: int)
    requires { x > 0 }
    ensures { exists t. t > 0 /\ x * t <> x }
=
    assert { forall u. u * 2 = u + u };
    ()

end
    \end{lstlisting}
    \end{frame}

    \begin{frame}{ghost-переменные}
    \begin{itemize}
    \item
    Если надо доказывать много утверждений про квантор существования,
    может быть проще завести переменную только для целей доказательства
    и нужным образом ее инициализировать или даже изменять прямо
    в функции. Такие переменные называются ghost-переменными.
    \item
    Синтаксис такой: \texttt{let ghost var = ... in ...}
    \end{itemize}
    \end{frame}

    \section{Модуль ref.Ref}

    \begin{frame}{Мотивация}
    \begin{itemize}
    \item
    Для каждого типа изменяемой переменной приходится писать
    тип с единственным mutable полем, причем имя поля должно
    быть уникально, это неудобно
    \item
    Можно решить проблему с уникальностью имени поля, сделав
    тип полиморфным, т.е. имеющим типовой параметр
    \item
    тем самым, уникальность будет обеспечивать
    значение типового параметра
    \end{itemize}
    \end{frame}

    \begin{frame}[fragile]{Полиморфный тип}
    Это может выглядеть примерно так:
    \begin{lstlisting}
type s (* only small letters! *)
type ref 't = { mutable contents: 't; }
let prog (x: int) (sv: s) =
    let y = { contents = 0 } in
        (* type of y is ref int *)
    let z = { contents = sv } in
        (* type of z is ref s *)
        ()
    \end{lstlisting}
    \end{frame}

    \begin{frame}{ref.Ref}
    \begin{itemize}
    \item
    В стандартной библиотеке why3 есть модуль \texttt{ref.Ref}.
    \item
    Он экспортирует такой же тип и функции для удобной работы
    с изменяемыми переменными.
    \item
    \texttt{let y = ref 0 in} -- создание переменной
    \item
    \texttt{y := !y + 1 } -- чтение переменной через
    восклицательный знак, изменение как в паскале
    \end{itemize}
    \end{frame}

\end{document}
