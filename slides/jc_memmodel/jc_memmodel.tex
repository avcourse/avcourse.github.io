\documentclass[hyperref={unicode=true}]{beamer}
\usepackage[utf8]{inputenx}
\usepackage[russian]{babel}

\usepackage{multicol}

\usepackage[pgf]{dot2texi}
\usepackage{tikz}
\usetikzlibrary{shapes, arrows}

\usepackage{listings}
\usepackage{graphicx}

\usepackage{comment}

\title{Лекция 10. Моделирование памяти Си-программы}
\author{}
\date{}

\usetheme{Warsaw}

\AtBeginSection[] {
	\begin{frame}{Содержание}
		\tableofcontents[currentsection]
	\end{frame}
}
%\overfullrule=5pt

\begin{document}
	\begin{frame}{}
		\titlepage
	\end{frame}

    \begin{frame}{Цель лекции}
    Построить аксиоматику для памяти Си-программы.
    \end{frame}

    \section{Память в стандарте языка Си}

    \begin{frame}{Указатели в языке Си}
    \begin{itemize}
    \item
    Указатель содержит адрес некоторого <<объекта>>.
    Операция разыменования (прочитать или
    заменить <<объект>>).
    \item
    Разыменование невыравненного указателя --
    неопределенное поведение (не может
    проконтролировать компилятор).
    \item
    Разыменование невалидного указателя --
    неопределенное поведение (не может
    проконтролировать компилятор).
    Валидный указатель -- тот, который можно
    корректно разыменовать.
    \end{itemize}
    \end{frame}

    \begin{frame}{Классы памяти}
    \begin{itemize}
    \item
    Статическая память -- распределяется до старта
    программы, существует все время работы
    (глобальные переменные всегда расположены в
     статической памяти, локальные переменные
     расположены в статической памяти, только если
     они объявлены со словом static)
    \item
    Стековая память -- распределяется при
    вызове функции, освобождается при возврате
    из функции
    \item
    Динамическая память -- распределяется в функциях
    \texttt{malloc}, \texttt{calloc}, \texttt{realloc},
    освобождается в функции \texttt{free}.
    \end{itemize}
    \end{frame}

    \begin{frame}{Адресная арифметика}
    \begin{itemize}
    \item
    Прибавление числа к указателю
    \item
    Вычитание указателей, сравнение указателей
    -- возможны только для указателей, указывающих
    на один блок памяти. Для остальных
    указателей -- неопределенное поведение.
    \item
    Указатели привязаны к блокам памяти!
    \item
    Блоки памяти не сравнимы!
    \item
    Блоки бывают освобожденными (freed).
    \item
    Приведение указателя к другому типу возможно,
    но результат стандартом не определяется.
    \end{itemize}
    \end{frame}

    \section{Модели памяти}

    \begin{frame}{Побайтовая модель памяти}
    \begin{itemize}
    \item Память -- это массив байтов, указатель --
    это индекс в массиве и размер типа объекта указателя.
    \item Как промоделировать освобождение блока?
    \item
    Как проконтролировать выход за границы одного блока,
    если за ним идет другой блок?
    \item Нужны дополнения
    в эту модель! Слишком сложно для аксиоматизации.
    \item Хотя и можно точно промоделировать конкретную
    Си-платформу.
    \end{itemize}
    \end{frame}

    \begin{frame}{Блочная модель памяти}
    \begin{itemize}
    \item Предположим, что не используется приведение
    типов указателей (и объединения), не сравниваются
    указатели из разных блоков памяти
    \item
    Указатель имеет блок и смещение от начала блока.
    \item
    Память -- это 1) множество блоков, у каждого блока есть
    число: если оно неотрицательное, то это размер блока,
    если оно отрицательное, значит этот блок освобожден.
    2) отображение из указателей в значения
    \item
    Указатели на объекты разных типов не лежат в одном блоке,
    поэтому для каждого типа объекта делаем свою память.
    \end{itemize}
    \end{frame}

    \begin{frame}{Оценка блочной модели памяти}
    \begin{itemize}
    \item Более простая аксиоматика
    \item Не моделируется ограниченность памяти
    \item Не моделируется класс памяти
    \item Не моделируется calloc и realloc
    (нужно байтовое представление для заполнения)
    \item Не моделируется malloc с
    размером, не кратным размеру объекта указателя
    \end{itemize}
    \end{frame}

    \begin{frame}{Нормализация кода}
    Все остальное моделируется в такой модели памяти?
    \begin{itemize}
    \item Операция взятия адреса: да, ее можно устранить:
    если есть \texttt{x} и \texttt{\&x}, то вместо \texttt{x}
    вводим новую переменную \texttt{xP}, которая хранит адрес \texttt{x};
    вместо \texttt{x} используем \texttt{*xP}, вместо
    \texttt{\&x} используем \texttt{P}.
    \item Переменные, на которые есть указатели: да,
    вместо них выделяем память при помощи \texttt{malloc},
    освобождаем при помощи \texttt{free}.
    \end{itemize}
    \end{frame}

    \section{Аксиоматизация блочной модели памяти}

    \begin{frame}[fragile]{Вспомогательная теория}
    \begin{lstlisting}
theory Map
  type map 'a 'b
  function get (m: map 'a 'b) (v: 'a): 'b
  function set (m: map 'a 'b) (v: 'a) (v2: 'b)
  axiom get_set: forall m: map 'a 'b,
      a a2: 'a, b: 'b.
      get (set m a b) a2 = (if a = a2 then b else
        get m a2)
end
    \end{lstlisting}
    \end{frame}

    \begin{frame}[fragile]{Основные символы}
    \begin{lstlisting}
type block
type pointer 't = (block, int)
type alloc_table 't = Map.map block int
type memory 't = Map.map (pointer 't) 't
    \end{lstlisting}
    \end{frame}

    \begin{frame}[fragile]{Дополнительные символы}
    \begin{lstlisting}
function blockOf (p: pointer 't): block
  = let (b, o) = p in b
function offsetOf (p: pointer 't): int
  = let (b, o) = p in o
function getSz (a: alloc_table 't) (b: block): int
  = Map.get a b
function setSz (a: alloc_table 't) (b: block)
 (sz: int): alloc_table 't = Map.set a b sz
function getV (m: memory 't) (p: pointer 't): 't
  = Map.get m p
function setV (m: memory 't) (p: pointer 't)
 (v: 't): memory 't = Map.set m p v
    \end{lstlisting}
    \end{frame}

    \begin{frame}[fragile]{Дополнительные символы}
    \begin{lstlisting}
function blockLen (a: alloc_table 't)
   (p: pointer 't): int = getSz a (blockOf p)
predicate valid (a: alloc_table 't) (p: pointer 't)
    =  0 <= offsetOf p < blockLen a p
predicate valid_range (a: alloc_table 't)
                      (p: pointer 't) (n: int) =
     0 <= offsetOf p < blockLen a p - n
predicate same_block (p p2: pointer 't) =
    blockOf p = blockOf p2
function shiftP (p: pointer 't)(n: int): pointer 't
 = let (b, o) = p in (b, o + n)
function subP (p p2: pointer 't): int
 = let (b, o) = p in let (b2, o2) = p2 in o - o2
    \end{lstlisting}
    \end{frame}
\iffalse
    \begin{frame}[fragile]{Аксиомы}
    \begin{lstlisting}
axiom get_set_alloc: forall a: alloc_table 't,
      b b2: block, sz: int.
      getSz (setSz a b sz) b2 = (if b = b2 then sz
              else getSz a b2)
axiom get_set_mem: forall m: memory 't,
      p p2: pointer 't, v: 't.
      getV (setV m p v) p2 = (if p = p2 then v
              else getV m p2)
    \end{lstlisting}
    \end{frame}
\fi
    \begin{frame}[fragile]{Примитивы для операторов CALL}
    \begin{lstlisting}
val readPtr (a: alloc_table 't) (m: memory 't)
            (p: pointer 't): 't
    requires { valid a p }
    ensures { result = getV m p }
val writePtr (a: alloc_table 't) (m: memory 't)
             (p: pointer 't) (v: 't): memory 't
    requires { valid a p }
    ensures { result = setV m p v }
val shiftPtr (p: pointer 't) (n: int): pointer 't
    ensures { same_block p result }
    ensures { offsetOf result = offsetOf p + n }
val subPtr (p: pointer 't) (p2: pointer 't): int
    requires { same_block p p2 }
    ensures { result = offsetOf p - offsetOf p2 }
    \end{lstlisting}
    \end{frame}

    \begin{frame}[fragile]{Примитивы (2)}
    \begin{lstlisting}
val malloc (m: memory 't): (memory 't, pointer 't)
    ensures { let (m2, p) = result in
        blockOf p was deallocated
        blockOf p was not used earlier (??????)
        blockOf p is allocated
        offsetOf p is 0
        sizes of other blocks are not changed
    }
    \end{lstlisting}
    \end{frame}

    \begin{frame}[fragile]{Примитивы (3)}
    \begin{lstlisting}
val free (m: memory 't) (p: pointer 't): memory 't
    requires { offset p is 0 }
    requires { blockOf p is allocated }
    ensures {
        blockOf p is deallocated
        sizes of other blocks are not changed }
    \end{lstlisting}
    \end{frame}

    \begin{frame}{Пример}
    Проверьте эффективность этой аксиоматизации
    модели памяти. Для этого решите следующие задания.
    \begin{itemize}
    \item Напишите на языке Си задачу прошлой лекции
    с массивом и заменой там одного значения на другое.
    Напишите модель программы. Напишите спецификацию.
    Докажите полную корректность модели программы
    относительно этой спецификации.
    \end{itemize}
    \end{frame}

    \begin{frame}{Вопросы}
    \begin{itemize}
    \item Как промоделировать \texttt{null}?
    \item Как промоделировать Си-структуры? Массивы
    из Си-структур?
    \item Как промоделировать неявное приведение
    \texttt{void *} к любому указателю?
    \end{itemize}
    \end{frame}
\end{document}
