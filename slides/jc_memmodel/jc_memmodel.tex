\documentclass[hyperref={unicode=true}]{beamer}
\usepackage[utf8]{inputenx}
\usepackage[russian]{babel}

\usepackage{multicol}

\usepackage[pgf]{dot2texi}
\usepackage{tikz}
\usetikzlibrary{shapes, arrows}

\usepackage{listings}
\usepackage{graphicx}

\usepackage{comment}

\title{Лекция 10. Моделирование памяти Си-программы}
\author{}
\date{}

\usetheme{Warsaw}

\AtBeginSection[] {
	\begin{frame}{Содержание}
		\tableofcontents[currentsection]
	\end{frame}
}
%\overfullrule=5pt

\begin{document}
	\begin{frame}{}
		\titlepage
	\end{frame}

    \begin{frame}{Цель лекции}
    Построить аксиоматику для памяти Си-программы.
    \end{frame}

    \section{Вспоминаем указатели}

    \begin{frame}[fragile]{Зачем нужны указатели?}
    Правильно ли здесь используются указатели?
    Что делает этот код?
    \begin{lstlisting}
void print__intar(const int *v, size_t n)
{
    for (int i = 0; i < n; ++i) {
        printf("%d ", v[i]);
    }
}
    \end{lstlisting}
    \end{frame}

    \begin{frame}[fragile]{Зачем нужны указатели?}
    Правильно ли здесь используются указатели?
    Что делает этот код?
    \begin{lstlisting}
int vector__init(vector *v, size_t n)
{
    v->capacity = v->size = n;
    v->data = malloc(n * sizeof *v);
    return v->data ? 0 : 1;
}
    \end{lstlisting}
    \end{frame}

    \begin{frame}[fragile]{Зачем нужны указатели?}
    Правильно ли здесь используются указатели?
    Что делает этот код?
    \begin{lstlisting}
int *a1 = malloc(10 * sizeof *a1);
int *a2 = (int *)malloc(10 * sizeof *a2);
    \end{lstlisting}
    \end{frame}

    \begin{frame}[fragile]{Зачем нужны указатели?}
    Правильно ли здесь используются указатели?
    Что делает этот код?
    \begin{lstlisting}
char *strcpy(char *d, const char *s)
{
    char *r = d;
    while (*r++ = *s++) ;
    return d;
}
    \end{lstlisting}
    \end{frame}

    \begin{frame}[fragile]{Зачем нужны указатели?}
    Правильно ли здесь используются указатели?
    Что делает этот код?
    \begin{lstlisting}
char *memmove(void *d, const void *s, size_t n)
{
    if (d + n < s || s + n < d) {
        return memcpy(d, s, n);
    } else {
        void *t = malloc(n);
        memcpy(t, s, n);
        memcpy(d, t, n);
        free(t);
        return d;
    }
}
    \end{lstlisting}
    \end{frame}

    \begin{frame}[fragile]{Зачем нужны указатели?}
    Правильно ли здесь используются указатели?
    Что делает этот код?
    \begin{lstlisting}
typedef struct node {
    struct node *next;
    int data;
} Node;
void push(Node **list, int v)
{
    Node *head = malloc(sizeof *head);
    head->next = *list;
    head->data = v;
    *list = head;
}
    \end{lstlisting}
    \end{frame}

    \begin{frame}[fragile]{Зачем нужны указатели?}
    Правильно ли здесь используются указатели?
    Что делает этот код?
    \begin{lstlisting}
...
int fd[2];
pipe(fd);
...
struct { int a, b; } msg = {1, 2};
write(fd[1], &msg, sizeof msg);
...
    \end{lstlisting}
    \end{frame}
\iffalse
    \begin{frame}[fragile]{Зачем нужны указатели?}
    Правильно ли здесь используются указатели?
    Что делает этот код?
    \begin{lstlisting}
...
int fd[2];
pipe(fd);
...
struct { short a; int b; } msg = {1, 2};
write(fd[1], &msg, sizeof msg);
...
    \end{lstlisting}
    \end{frame}

    \begin{frame}[fragile]{Зачем нужны указатели?}
    Правильно ли здесь используются указатели?
    Что делает этот код?
    \begin{lstlisting}
...
int fd[2];
char buf[1024];
pipe(fd);
...
struct { short a; int b; } msg;
read(fd[0], &buf, sizeof buf);
char n = buf[0];
msg = *(msg *)(buf + 1);
printf("%s %d\n", msg.a, msg.b);
...
    \end{lstlisting}
    \end{frame}

    \begin{frame}[fragile]{Зачем нужны указатели?}
    Правильно ли здесь используются указатели?
    Что делает этот код?
    \begin{lstlisting}
int anint;
void foo(long *lptr)
{
    anint=1;
    *lptr=3;
    printf("%d", anint);
}
... foo((long *)&anint);
    \end{lstlisting}
Печать зависит от опций компилятора...
    \end{frame}

    \begin{frame}[fragile]{Зачем нужны указатели?}
    Правильно ли здесь используются указатели?
    Что делает этот код?
    \begin{lstlisting}
uint32_t swaphalves(uint32_t a)
{
    uint32_t acopy = a;
    uint16_t *ptr = (uint16_t*)&acopy;
    uint16_t tmp = ptr[0];
    ptr[0] = ptr[1];
    ptr[1] = tmp;
    return acopy;
}
    \end{lstlisting}
    \end{frame}

    \begin{frame}[fragile]{Зачем нужны указатели?}
    Правильно ли здесь используются указатели?
    Что делает этот код?
    \begin{lstlisting}
uint32_t swaphalves(uint32_t a)
{
    union {
        uint32_t as32bit;
        uint16_t as16bit[2];
    } s = { a };
    uint16_t tmp = s.as16bit[0];
    s.as16bit[0] = s.as16bit[1];
    s.as16bit[1] = tmp;
    return s.as32bit;
}
\end{lstlisting}
    \end{frame}

    \begin{frame}[fragile]{Зачем нужны указатели?}
    Правильно ли здесь используются указатели?
    Что делает этот код?
    \begin{lstlisting}
struct sockaddr_in sa = {0};
int sockfd = ...;
sa.sin_family = AF_INET;
sa.sin_port = htons(port);
bind(sockfd, (struct sockaddr *)&sa, sizeof sa);
    \end{lstlisting}
    \end{frame}
\fi
    \section{Вырабатываем интуицию в модели памяти}

    \begin{frame}{Примеры}
    Попробуйте придумать алгоритм построения модели
    программы, подходящий для всех следующих примеров.
    \end{frame}

    \begin{frame}[fragile]{Пример 1}
    \begin{lstlisting}
void f1(int *a, int n) { ... }
void f() {
    int mm[10];
    f1(&mm[3], 4);
}
    \end{lstlisting}
    Всегда ли параметр-указатель означает начало массива?
    \end{frame}

    \begin{frame}[fragile]{Пример 2}
    \begin{lstlisting}
void f2(int *a, int n, int *v) { ... }
    \end{lstlisting}
    Всегда ли параметр-указатель означает, что функция
    получает массив (отдельный блок памяти)?
    \end{frame}

    \begin{frame}[fragile]{Пример 3}
    \begin{lstlisting}
void f3(int *a1, int n1, int *a2, int n2) { ... }
    \end{lstlisting}
    Всегда ли эта функция получает два массива
    (отдельных блока памяти)?
    \end{frame}

    \begin{frame}[fragile]{Пример 4}
    \begin{lstlisting}
int Vector__resize(struct Vector *v, int newsize) {
    ... realloc ...
}
    \end{lstlisting}
    Как записать постусловие, что эта функция может
    аллоцировать новый блок памяти? (1 - что блок
    аллоцируется, 2 - что блока не было раньше)
    Какие входные данные нужны, чтобы записать это
    постусловие?
    \end{frame}

    \begin{frame}[fragile]{Пример 5}
    \begin{lstlisting}
int main(int argc, char *argv[]) { ... }
    \end{lstlisting}
    Сколько массивов (отдельных блоков памяти) нужно
    передать как входные данные этой функции?
    \end{frame}

    \begin{frame}{Выводы из примеров}
    \begin{itemize}
    \item
    Кроме своих аргументов, функции получают неявно
    еще входные данные - блоки памяти.
    \item
    Количество блоков памяти иногда трудно, а иногда
    невозможно определить по тексту программы.
    \end{itemize}
    \end{frame}

\iffalse
    \begin{frame}{Указатели в языке Си}
    \begin{itemize}
    \item
    Указатель содержит адрес некоторого <<объекта>>.
    \item
    Операция разыменования (прочитать или
    заменить <<объект>>), функции malloc, free
    \item
    Ошибки при работе с указателями:
        \begin{itemize}
        \item разыменование невыравненного указателя
        \item разыменование невалидного указателя
        (указатель в еще не выделенную область памяти
        или в уже освобожденную область памяти)
        \item сравнение указателей из разных <<объектов>>
        \item наличие указателей несовместимых типов
        на один и тот же <<объект>>
        \end{itemize}
    \end{itemize}
    \end{frame}

    \begin{frame}{Классы памяти}
    \begin{itemize}
    \item
    Статическая память -- распределяется до старта
    программы, существует все время работы
    (глобальные переменные всегда расположены в
     статической памяти, локальные переменные
     расположены в статической памяти, только если
     они объявлены со словом static)
    \item
    Стековая память -- распределяется при
    вызове функции, освобождается при возврате
    из функции
    \item
    Динамическая память -- распределяется в функциях
    \texttt{malloc}, \texttt{calloc}, \texttt{realloc},
    освобождается в функции \texttt{free}.
    \end{itemize}
    \end{frame}

    \begin{frame}{Адресная арифметика}
    \begin{itemize}
    \item
    Прибавление числа к указателю
    \item
    Вычитание указателей, сравнение указателей
    -- возможны только для указателей, указывающих
    на один блок памяти. Для остальных
    указателей -- неопределенное поведение.
    \item
    Указатели привязаны к блокам памяти!
    \item
    Блоки памяти не сравнимы!
    \item
    Блоки бывают освобожденными (freed).
    \item
    Приведение указателя к другому типу указателя возможно
    только если старый и новый типы совместимы.
    \end{itemize}
    \end{frame}
\fi
    \section{Модели памяти}

\iffalse
    \begin{frame}{Побайтовая модель памяти}
    \begin{itemize}
    \item Память -- это массив байтов, указатель --
    это индекс в массиве и размер типа объекта указателя.
    \item Как промоделировать освобождение блока?
    \item
    Как проконтролировать выход за границы одного блока,
    если за ним идет другой блок?
    \item Нужны дополнения
    в эту модель! Слишком сложно для аксиоматизации.
    \item Хотя и можно точно промоделировать конкретную
    Си-платформу.
    \end{itemize}
    \end{frame}

    \begin{frame}{Блочная модель памяти}
    \begin{itemize}
    \item Предположим, что не используется приведение
    типов указателей (и объединения), не сравниваются
    указатели из разных блоков памяти
    \item
    Указатель имеет блок и смещение от начала блока.
    \item
    Память -- это 1) множество блоков, у каждого блока есть
    число: если оно неотрицательное, то это размер блока,
    если оно отрицательное, значит этот блок освобожден.
    2) отображение из указателей в значения
    \item
    Указатели на объекты разных типов не лежат в одном блоке,
    поэтому для каждого типа объекта делаем свою память.
    \end{itemize}
    \end{frame}

    \begin{frame}{Оценка блочной модели памяти}
    \begin{itemize}
    \item Более простая аксиоматика
    \item Не моделируется ограниченность памяти
    \item Не моделируется класс памяти
    \item Не моделируется calloc и realloc
    (нужно байтовое представление для заполнения)
    \item Не моделируется malloc с
    размером, не кратным размеру объекта указателя
    \end{itemize}
    \end{frame}

    \begin{frame}{Нормализация кода}
    Все остальное моделируется в такой модели памяти?
    \begin{itemize}
    \item Операция взятия адреса: да, ее можно устранить:
    если есть \texttt{x} и \texttt{\&x}, то вместо \texttt{x}
    вводим новую переменную \texttt{xP}, которая хранит адрес \texttt{x};
    вместо \texttt{x} используем \texttt{*xP}, вместо
    \texttt{\&x} используем \texttt{P}.
    \item Переменные, на которые есть указатели: да,
    вместо них выделяем память при помощи \texttt{malloc},
    освобождаем при помощи \texttt{free}.
    \end{itemize}
    \end{frame}

    \section{Аксиоматизация блочной модели памяти}
\fi

    \begin{frame}{Состав модели памяти}
    \begin{itemize}
    \item
    массив -- типизированный, поэтому блок -- тоже типизированный
    \item
    поэтому для каждого типа нужна целиком отдельная модель памяти
    \item
    тип-символ memory - вся модель памяти
    \item
    тип-символ block - блок
    \item
    тип-символ pointer - указатель
    \end{itemize}
    \end{frame}

    \begin{frame}{Соотношения}
    \begin{itemize}
    \item
    каждый memory -- множество из block-ов
    \item
    каждый block имеет массив array
    \item
    любой block может быть allocated или не allocated (allocable)
    \item
    любой pointer имеет block и индекс в его массиве
    \item
    любой pointer может быть равен NULL
    \item
    любые два pointer из разных block-ов не сравнимы
    \end{itemize}
    \end{frame}

    \begin{frame}{Предикатные и функциональные символы, примитивы}
    \begin{itemize}
    \item
    любой pointer является valid (валидным, корректно разыменуемым), если
    он указывает внутрь block, который allocated
    \item
    адресная арифметика: shift pointer, sub pointers
    \item
    примитивы для блоков: malloc, free
    \item
    примитивы для указателей (разыменование): write, read
    \end{itemize}
    \end{frame}

    \begin{frame}{Первые определения - первые проблемы}
    \begin{itemize}
    \item
    Попробуйте определить спецификацию для read - должна нормально получиться
    \item
    Попробуйте определить спецификацию для write - получится слишком
    много кванторов! Как их уменьшить? Переделать модель памяти.
    \item
    Надо иметь всего один массив (для значений по указателям).
    Как тогда определить индексы и адреса начала блока? Массив надо заменить
    просто на отображение указателей в значения.
    \item
    И отделить множество блоков от этого отображения. Получаются 2 отдельных
    типа, совместно представляющих модель памяти: отображение и множество блоков
    с их статусом.
    \end{itemize}
    \end{frame}

    \begin{frame}[fragile]{Определяем отображения}
    \begin{lstlisting}
theory Map
  type map 'a 'b
  function get (m: map 'a 'b) (v: 'a): 'b
  function set (m: map 'a 'b) (v: 'a) (v2: 'b)
  axiom get_set: forall m: map 'a 'b,
      a a2: 'a, b: 'b.
      get (set m a b) a2 = (if a = a2 then b else
        get m a2)
end
    \end{lstlisting}
    \end{frame}

    \begin{frame}[fragile]{Типы}
    \begin{lstlisting}
type block
type pointer 't = (block, int)
type alloc_table 't = Map.map block int
(* alloc_table[block] >= 0 -- block is allocated
   and here is its size;
   alloc_table[block] < 0 -- block is allocable *)
type memory 't = Map.map (pointer 't) 't
    \end{lstlisting}
    \end{frame}

    \begin{frame}[fragile]{Дополнительные символы}
    \begin{lstlisting}
function blockOf (p: pointer 't): block
  = let (b, o) = p in b
function offsetOf (p: pointer 't): int
  = let (b, o) = p in o
function getSz (a: alloc_table 't) (b: block): int
  = Map.get a b
function setSz (a: alloc_table 't) (b: block)
 (sz: int): alloc_table 't = Map.set a b sz
function select (m: memory 't) (p: pointer 't): 't
  = Map.get m p
function store (m: memory 't) (p: pointer 't)
 (v: 't): memory 't = Map.set m p v
    \end{lstlisting}
    \end{frame}

    \begin{frame}[fragile]{Дополнительные символы}
    \begin{lstlisting}
function blockLen (a: alloc_table 't)
   (p: pointer 't): int = getSz a (blockOf p)
predicate allocable (a: alloc_table 't)
                    (p: pointer 't)
    = blockLen a p < 0
predicate allocated (a: alloc_table 't)
                    (p: pointer 't)
    = not (allocable a p)
predicate freeable (a: alloc_table 't)
                   (p: pointer 't)
    = allocated a p /\ offsetOf p = 0
    \end{lstlisting}
    \end{frame}

    \begin{frame}[fragile]{Дополнительные символы}
    \begin{lstlisting}
predicate valid (a: alloc_table 't) (p: pointer 't)
    =  0 <= offsetOf p < blockLen a p
predicate valid_range (a: alloc_table 't)
                      (p: pointer 't) (n: int) =
     0 <= offsetOf p < blockLen a p - n
predicate same_block (p p2: pointer 't) =
    blockOf p = blockOf p2
function shiftP (p: pointer 't)(n: int): pointer 't
 = let (b, o) = p in (b, o + n)
function subP (p p2: pointer 't): int
 = let (b, o) = p in let (b2, o2) = p2 in o - o2
    \end{lstlisting}
    \end{frame}

    \begin{frame}[fragile]{Примитивы для операторов CALL}
    \begin{lstlisting}
val readPtr (a: alloc_table 't) (m: memory 't)
            (p: pointer 't): 't
    requires { valid a p }
    ensures { result = select m p }
val writePtr (a: alloc_table 't) (m: memory 't)
             (p: pointer 't) (v: 't): memory 't
    requires { valid a p }
    ensures { result = store m p v }
val shiftPtr (p: pointer 't) (n: int): pointer 't
    ensures { same_block p result }
    ensures { offsetOf result = offsetOf p + n }
val subPtr (p: pointer 't) (p2: pointer 't): int
    requires { same_block p p2 }
    ensures { result = offsetOf p - offsetOf p2 }
    \end{lstlisting}
    \end{frame}

    \begin{frame}[fragile]{Примитивы (2)}
    \begin{lstlisting}
val malloc (a: alloc_table 't) (m: memory 't)
    (sz: int): (alloc_table 't, pointer 't)
    requires { sz >= 0 }
    ensures { let (a2, p) = result in
        blockOf p was allocable
        p is freeable
        blockLen p = sz
        sizes of other blocks are not changed
        allocation statuses of other blocks
              are not changed
    }
    \end{lstlisting}
    \end{frame}

    \begin{frame}[fragile]{Примитивы (3)}
    \begin{lstlisting}
val free (a: alloc_table 't) (m: memory 't)
    (p: pointer 't): alloc_table 't
    requires { p is freeable }
    ensures {
        blockOf p is allocable
        sizes of other blocks are not changed
        allocation statuses of other blocks
              are not changed
    }
    \end{lstlisting}
    \end{frame}

    \begin{frame}{Пример}
    Проверьте эффективность этой аксиоматизации
    модели памяти. Для этого решите следующие задания.
    \begin{itemize}
    \item Напишите на языке Си задачу прошлой лекции
    с массивом и заменой там одного значения на другое.
    Напишите модель программы. Напишите спецификацию.
    Докажите полную корректность модели программы
    относительно этой спецификации.
    \end{itemize}
    \end{frame}

    \begin{frame}{Вопросы}
    \begin{itemize}
    \item Как промоделировать \texttt{null}?
    \item Как промоделировать Си-структуры? Массивы
    из Си-структур?
    \item Как промоделировать неявное приведение
    \texttt{void *} к любому указателю?
    \end{itemize}
    \end{frame}
\end{document}
