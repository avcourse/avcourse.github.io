\documentclass[hyperref={unicode=true}]{beamer}
\usepackage[utf8]{inputenx}
\usepackage[russian]{babel}

\usepackage{multicol}

\usepackage[pgf]{dot2texi}
\usepackage{tikz}
\usetikzlibrary{shapes, arrows}

\usepackage{listings}
\usepackage{graphicx}

\usepackage{comment}

\title{Лекция 4. Методы Флойда для блок-схем с вызовами других блок-схем}
\author{}
\date{}

\usetheme{Warsaw}

\AtBeginSection[] {
	\begin{frame}{Содержание}
		\tableofcontents[currentsection]
	\end{frame}
}
%\overfullrule=5pt

\begin{document}
	\begin{frame}{}
		\titlepage
	\end{frame}

    \begin{frame}{Цель лекции}
    Добавить в блок-схемы возможность вызвать
    другую блок-схему как подпрограмму.
    Определить методы Флойда для таких блок-схем.
    \end{frame}

    \section{Синтаксис и семантика}

    \begin{frame}{Подпрограммы в языках программирования}
    \begin{itemize}
    \item подпрограмма -- это модуль (белый ящик, черный ящик),
            заголовок, прототип, сигнатура, интерфейс и реализация
    \item процедурная абстракция -- годится любая реализация
            вызываемой подпрограммы
            из определенного множества реализаций
    \item формальные и фактические параметры
    \item способы передачи параметров в функцию (по значению,
            по ссылке, по имени)
    \item требуется этап сборки программы из подпрограмм
    \end{itemize}
    \end{frame}

	\begin{frame}[fragile]{Пример блок-схемы}

	\huge
	\begin{dot2tex}[options=-traw]
	digraph G{
		d2tgraphstyle="scale=0.4, transform shape";

		/* nodes by levels */
		node[shape=rectangle, height=1];
		START[style=rounded, width=2, texlbl="$\begin{matrix}START:\\(y_1,~y_2) \leftarrow (0,~0)\end{matrix}$"];
        COND[style=rounded, width=2.5, texlbl="$x_1 \geq 0$"];
		CALL1[width=4, texlbl="$\begin{matrix}CALL:\\(y_1,~y_2) \leftarrow DIV(x_1,~x_2)\end{matrix}$"];
		CALL2[width=4, texlbl="$\begin{matrix}CALL:\\(y_1,~y_2) \leftarrow DIV(-x_1,~x_2)\end{matrix}$"];
		ASSIGN[width=4, texlbl="$(y_1,~y_2) \leftarrow (-y_1-1,~x_2 - y_2)$"];
        HALT[style=rounded, width=4, texlbl="$\begin{matrix}HALT:\\(z_1,~z_2) \leftarrow (y_1,~y_2)\end{matrix}$"];

		/* edges */
		node [shape=point, width=0, label=""];
		START -> COND [weight=8];
		{ rank=same; p3 -> COND [label="T", arrowhead=none]; COND -> p4 [label="F", arrowhead=none]; }
		p3 -> CALL1 [weight=8];
		p4 -> CALL2 [weight=8];
		{ rank=same; CALL1; CALL2; }
		CALL1 -> HALT [weight=8];
		CALL2 -> ASSIGN [weight=8];
		{ rank=same; rankdir="RL"; HALT -> ASSIGN; }
        }
	\end{dot2tex}

	\normalsize

    $D_{x_1} = D_{x_2} = D_{y_1} = D_{y_2} = D_{z_1} = D_{z_2} = \mathbb{Z}$.
    Как определить функцию, вычисляемую этой блок-схемой?
	\end{frame}

    \begin{frame}{Синтаксис и семантика}
    \begin{itemize}
    \item синтаксис: сигнатура блок-схемы, домены ее входных переменных и
            домены ее выходных переменных
    \item пример: сигнатура $DIV$ с
          доменом входных переменных $\mathbb{Z} \times \mathbb{Z}$ и
          доменом выходных переменных $\mathbb{Z} \times \mathbb{Z}$
    \item вычисление требует предъявления блок-схемы, подходящей под сигнатуру
    \item т.е. функция, вычисляемая блок-схемой, \emph{зависит}
          от функции, вычисляемой вызываемыми блок-схемами
    \end{itemize}
    \end{frame}

    \begin{frame}[fragile]{Пример функции, вычисляемой блок-схемой}
    \begin{table}[]
    \begin{tabular}{|c|c|}
    \hline
    $M[DIV](a_1,~a_2) =$ & $M[P](x_1,~x_2) =$ \\
    \hline
    $(0,~0)$
    &
    $\begin{cases}
            (0,~0) &, x_1 \geq 0 \\
            (-1,~x_2) &, x_1 < 0\end{cases}$\\
    \hline
    $\begin{cases}
            (0,~0) &, a_1 \geq 0 \\
            (-1,~a_2) &, a_1 < 0\end{cases}$
    &
    $\begin{cases}
            (0,~0) &, x_1 \geq 0 \\
            (-1,~x_2) &, x_1 < 0\end{cases}$ \\
    \hline
    $\begin{cases}
            \omega &, a_1 \geq 0 \\
            (0, 0) &, a_1 < 0 \\ \end{cases}$
    &
    $\omega$ \\
    \hline
    $\begin{cases}
            (a_1~/~a_2,~a_1~\%~a_2) &, a_1 \geq 0 \land a_2 > 0 \\
            \omega &, \mbox{иначе}\end{cases}$
    &
    ?\\ \hline
    \end{tabular}
    \end{table}
    \end{frame}

    \begin{frame}{Модель требований для блок-схем с вызовами}
    не требует изменений по сравнению с блок-схемами без вызовов
    \end{frame}

    \section{Полная корректность для нерекурсивного случая}

    \begin{frame}{Соотношения корректности: рекурсии пока нет}
    \begin{itemize}
    \item Хотим модульности -- доказательство корректности не надо
    переделывать при смене вызываемой блок-схемы, если она из
    <<нужного>> множества.
    \item Множество блок-схем, полностью корректных относительно
          некоторой спецификации, вот пример <<нужного>> множества
    \item Можно задать это множество при помощи пары предикатов --
            предусловия и постусловия (спецификации сигнатуры)
    \item Блок-схема с вызовами полностью корректна относительно
            своей спецификации и спецификаций сигнатур, если
            для всех блок-схем, полностью корректных относительно
            спецификаций сигнатур, ... (далее определение полной
                    корректности для блок-схем без вызовов)
    \end{itemize}
    \end{frame}

    \begin{frame}{Формулы для полной корректности}
    \begin{itemize}
    \item Смотрим на наш пример с DIV. Пусть $(\varphi_D,~\psi_D)$ --
        спецификация, сопоставленная сигнатуре DIV.
    \item Полная корректность = частичная корректность + завершаемость
    \item Завершаемость: если CALL с сигнатурой DIV зацикливается,
                то зацикливается и Р. Пробуем выразить определение
                полной корректности (завершаемости) без квантора по
                функциям, вычисляемым блок-схемами, которые сопоставлены DIV.
    \item Какова <<самая зацикливающаяся>> блок-схема, полностью
            корректная относительно $(\varphi_D,~\psi_D)$? Та, которая
            зацикливается на всех входах, где ложно $\varphi_D$, и
            завершается на всех входах, где истинно $\varphi_D$.
    \item Значит, Р завершается тогда, когда завершается DIV, т.е.
            входы DIV удовлетворяют $\varphi_D$.
    \end{itemize}
    \end{frame}

    \begin{frame}{Формулы для полной корректности}
    \begin{itemize}
    \item Частичная корректность: если DIV завершилась, то ее выходные
            переменные удовлетворяют постусловию $\psi_D$. Надо доказать,
            что при всех таких значениях переменных вычисления будут
            приводить к HALT в блок-схеме Р с теми переменными, которые
            удовлетворяют постусловию $\psi$.
    \item Квантор по функциям, вычисляемым блок-схемами, которые
            сопоставлены DIV, превращается в квантор по разным значениям
            выходных переменных DIV, удовлетворяющих постусловию $\psi_D$.
    \end{itemize}
    \end{frame}

    \begin{frame}{Формулы в примере}
    Получаются следующие формулы для доказательства полной корректности
    примера:

    \begin{itemize}
    \item завершаемость вызова DIV (путь START-T):
        $\forall x_1, x_2 \in \mathbb{Z} \cdot
                \varphi(x_1,~x_2) \land x_1 \geq 0
                \Rightarrow \varphi_D(x_1,~x_2)$
    \item завершаемость вызова DIV (путь START-F):
        $\forall x_1, x_2 \in \mathbb{Z} \cdot
                \varphi(x_1,~x_2) \land x_1 < 0
                \Rightarrow \varphi_D(-x_1,~x_2)$
    \item частичная корректность на пути (START-T-HALT):
        $\forall x_1, x_2 \in \mathbb{Z} \forall r_1, r_2 \in \mathbb{Z} \cdot
                \varphi(x_1,~x_2) \land x_1 \geq 0 \land
                \psi_D(x_1,~x_2,~r_1,~r_2)
                \Rightarrow \psi(x_1,~x_2,~r_1,~r_2)$
    \item частичная корректность на пути (START-F-HALT):
        $\forall x_1, x_2 \in \mathbb{Z} \forall r_1, r_2 \in \mathbb{Z} \cdot
                \varphi(x_1,~x_2) \land x_1 < 0 \land
                \psi_D(-x_1,~x_2,~r_1,~r_2)
                \Rightarrow \psi(x_1,~x_2,~-r_1-1,~x_2 - r_2)$
    \end{itemize}
    \end{frame}

    \begin{frame}{Формулы для блок-схем с циклами и вызовами}
    Составляем формулы по тому же принципу, но отсчитываем
    базовые пути не только от START, но и от каждой точки сечения.
    \end{frame}

    \section{Полная корректность для рекурсии}

    \begin{frame}{Рекурсия}
    \begin{itemize}
    \item Определение полной корректности такое же, как и для блок-схем
            без вызовов (не требуется выбирать <<нужное>> множество).
    \item Доказываем полную корректность по индукции по глубине рекурсии.
    \item Базовый случай: глубина рекурсии равна 0 (выход из рекурсии).
    \item Индуктивный переход: считаем, что доказали полную корректность
    на тех значениях входных переменных, которые приводят к глубине рекурсии
    не больше $n$. Тогда если все вызовы блок-схем такие, то можно
    сделать индуктивный переход -- доказать полную корректность
    для множества значений входных переменных, приводящих к глубине рекурсии
    $n + 1$.
    \end{itemize}
    \end{frame}

    \begin{frame}{Завершаемость индукции}
    \begin{itemize}
    \item Надо доказать, что все рекурсивные вызовы приводят к
            меньшей глубине рекурсии.
    \item Можно формализовать это правило при помощи фундированного множества
            и оценочной функции.
    \item Оценочная функция сопоставляется блок-схеме. Ее область определения
            -- входной домен блок-схемы. Ее область значений -- фундированное
            множество.
    \end{itemize}
    \end{frame}

    \begin{frame}{Формулы завершаемости}
    \begin{itemize}
    \item Надо доказать фундированность множества.
    \item Надо доказать, что при всяком вызове блок-схемы
            с входными переменными, удовлетворяющими предусловию,
            значение оценочной функции принадлежит фундированному множеству.
    \item Надо доказать, что при всяком базовом пути, приводящем к
            рекурсивному вызову, оценочная функция на входных переменных
            вызываемой блок-схемы меньше оценочной функции на входных
            переменных вызывающей блок-схемы.
    \end{itemize}
    \end{frame}

    \begin{frame}{Формулы полной корректности}
    \begin{itemize}
    \item Как только доказана завершаемость индукции, ее можно применять
    для доказательства частичной корректности и завершаемости блок-схемы.
    \item Применяем формулы для доказательства полной корректности
        для нерекурсивного случая, сопоставив вызываемым блок-схемам
        спецификацию вызывающей блок-схемы.
    \end{itemize}
    \end{frame}

    \begin{frame}[fragile]{Пример}
    Вычисление частного и остатка при делении
	\huge
	\begin{dot2tex}[options=-traw]
	digraph G{
		d2tgraphstyle="scale=0.4, transform shape";

		/* nodes by levels */
		node[shape=rectangle, height=1];
		START[style=rounded, width=2, texlbl="$\begin{matrix}START:\\(y_1,~y_2) \leftarrow (0,~0)\end{matrix}$"];
        COND[style=rounded, width=2.5, texlbl="$x_1 < x_2$"];
		HALT1[style=rounded, width=4, texlbl="$\begin{matrix}HALT:\\(z_1,~z_2) \leftarrow (0,~x_1)\end{matrix}$"];
		CALL[width=4, texlbl="$\begin{matrix}CALL:\\(y_1,~y_2) \leftarrow P(x_1-x_2,~x_2)\end{matrix}$"];
        HALT2[style=rounded, width=4, texlbl="$\begin{matrix}HALT:\\(z_1,~z_2) \leftarrow (y_1+1,~y_2)\end{matrix}$"];

		/* edges */
		node [shape=point, width=0, label=""];
		START -> COND [weight=8];
		{ rank=same; p3 -> COND [label="T", arrowhead=none]; COND -> p4 [label="F", arrowhead=none]; }
		p3 -> HALT1 [weight=8];
		p4 -> CALL [weight=8];
		{ rank=same; HALT1; CALL; }
		CALL -> HALT2 [weight=8];
        }
	\end{dot2tex}

	\normalsize

    $D_{x_1} = D_{x_2} = D_{y_1} = D_{y_2} = D_{z_1} = D_{z_2} = \mathbb{Z}$.
	    \end{frame}
% появится ли где-то, что <<нужное>> множество должно быть непусто?

    \begin{frame}{Пример::схема доказательства}
    \begin{enumerate}
    \item
    Надо доказать полную корректность блок-схемы Р относительно
    спецификации: $\varphi(x_1,~x_2) = x_1 \geq 0 \land x_2 > 0$,
    $\psi(x_1,~x_2,~z_1,~z_2) = (x_1 = z_1 * x_2 + z_2 \land 0 \leq z_2 < x_2)$.

    \item
    Доказываем возможность индукции по глубине рекурсии:
    берем фундированное множество $(\{0, 1, 2, ...\}, <)$. Его фундированность
    была доказана ранее. Берем оценочную функцию $u(x_1,~x_2) = x_1$.

    \item
    Составляем условия верификации.
    \end{enumerate}
    \end{frame}

    \begin{frame}{Пример::условия верификации}
    \begin{itemize}
    \item корректность оценочной функции:
            $\forall x_1,~x_2 \in \mathbb{Z} \cdot
            \varphi(x_1,~x_2)
            \Rightarrow
            u(x_1,~x_2) \geq 0$
    \item завершаемость индукции:
            $\forall x_1,~x_2 \in \mathbb{Z} \cdot
            \varphi(x_1,~x_2) \land x_1 \geq x_2
            \Rightarrow
            \varphi(x_1 - x_2,~x_2)$
    \item частичная корректность (путь START-T-HALT):
            $\forall x_1,~x_2 \in \mathbb{Z} \cdot
            \varphi(x_1,~x_2) \land x_1 < x_2
            \Rightarrow
            \psi(x_1,~x_2,~0,~x_1)$
    \item частичная корректность (путь START-F-HALT):
            $\forall x_1,~x_2,~y_1,~y_2 \in \mathbb{Z} \cdot
            \varphi(x_1,~x_2) \land x_1 \geq x_2 \land
            \psi(x_1 - x_2,~x_2,~y_1,~y_2)
            \Rightarrow
            \psi(x_1,~x_2,~y_1 + 1,~y_2)$
    \item завершаемость: не требует условий верификации, т.к. нет циклов
    \end{itemize}
    \end{frame}

    \begin{frame}{Подставляем формулы}
    \begin{itemize}
    \item корректность оценочной функции:
            $\forall x_1,~x_2 \in \mathbb{Z} \cdot
            x_1 \geq 0 \land x_2 > 0
            \Rightarrow
            x_1 \geq 0$
    \item завершаемость индукции:
            $\forall x_1,~x_2 \in \mathbb{Z} \cdot
            x_1 \geq 0 \land x_2 > 0 \land x_1 \geq x_2
            \Rightarrow
            x_1 - x_2 \geq 0 \land x_2 > 0$
    \item частичная корректность (путь START-T-HALT):
            $\forall x_1,~x_2 \in \mathbb{Z} \cdot
            x_1 \geq 0 \land x_2 > 0 \land x_1 < x_2
            \Rightarrow
            x_1 = x_1 + 0 * x_2 \land 0 \leq x_1 < x_2$
    \item частичная корректность (путь START-F-HALT):
            $\forall x_1,~x_2,~y_1,~y_2 \in \mathbb{Z} \cdot
            x_1 \geq 0 \land x_2 > 0 \land x_1 \geq x_2 \land
            x_1 - x_2 = x_2 * y_1 + y_2 \land 0 \leq y_2 < x_2
            \Rightarrow
            x_1 = x_2 * (y_1 + 1) + y_2 \land 0 \leq y_2 < x_2$
    \end{itemize}
    \end{frame}

\end{document}

