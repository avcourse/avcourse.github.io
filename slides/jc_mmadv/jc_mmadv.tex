\documentclass[hyperref={unicode=true}]{beamer}
\usepackage[utf8]{inputenx}
\usepackage[russian]{babel}

\usepackage{multicol}

\usepackage[pgf]{dot2texi}
\usepackage{tikz}
\usetikzlibrary{shapes, arrows}

\usepackage{listings}
\usepackage{graphicx}

\usepackage{comment}

\title{Лекция 11. Фрейм функции. Component-as-array.
    Более простая аксиоматизация модели памяти.}
\author{}
\date{}

\usetheme{Warsaw}

\AtBeginSection[] {
	\begin{frame}{Содержание}
		\tableofcontents[currentsection]
	\end{frame}
}
%\overfullrule=5pt

\begin{document}
	\begin{frame}{}
		\titlepage
	\end{frame}

    \begin{frame}{Цель лекции}
    Обсудить ряд важных дополнительных вопросов
    про аксиоматизацию блочной модели памяти.
    \end{frame}

    \section{Фрейм функции (footprint)}

    \begin{frame}{Мотивация}
    \begin{itemize}
    \item
    Вы уже написали функцию, заменяющую
    одно значение на другое значение в массиве.
    \item
    Напишите еще одну функцию, чтобы проверить
    спецификацию первой функции. Функция принимает
    на вход массив (указатель на начало и размер)
    и 3 значения \texttt{v0, v1, v2}. Размер массива
    не должен быть меньше 2. Постусловие функции --
    что 0-й элемент массива становится равным
    \texttt{v0}, а 1-й элемент массива становится
    равным \texttt{v2}. Функция сначала присваивает
    в 0-й элемент массива значение \texttt{v0},
    в 1-й элемент массива значение \texttt{v1},
    затем вызывает первую функцию для части массива,
    с 1-го элемента до его конца.
    \item
    Почему одно из условий постусловия не доказывается?
    \end{itemize}
    \end{frame}

    \begin{frame}{Фрейм функции}
    \begin{itemize}
    \item
    В спецификации первой функции не сказано, что
    может делать функция за рамками переданного массива.
    Значит, она может делать что угодно.
    \item
    Необходимо думать о границах области памяти (рамке,
    фрейме, footprint), которую разрешено изменять функции.
    \item
    Добавьте соответствующее постусловие к первой функции.
    \item
    Проверьте, что если второй функции передать еще один массив
    и с ним вторая функция ничего не делает, то этот факт
    доказывается!
    \end{itemize}
    \end{frame}

    \section{Component-as-array}

    \begin{frame}{Цель}
    \begin{itemize}
    \item Как моделировать Си-структуры? Массивы Си-структур?
    \item Первая идея -- использовать WhyML-структуры один-в-один,
    те же поля, что и у Си-структур
    \item В модели памяти блоки будут состоять из указателей на
    структуру целиком
    \item Можно улучшить этот принцип моделирования
    \end{itemize}
    \end{frame}

    \begin{frame}{Component-as-array}
    \begin{itemize}
    \item Отдельные поля одной структуры не могут
    находиться в одной области памяти (это не union).
    \item Создадим отдельный \texttt{memory map} для
    каждого поля структуры.
    \item (+) Упрощение верификации
    \item (-) Усложнение генерации модели программы (нужна
            автоматизация)
    \end{itemize}
    \end{frame}

    \section{Упрощение аксиоматизации модели памяти}

    \begin{frame}{Убираем тип block}
    \begin{itemize}
    \item
    Тип \texttt{block} отличается от других тем, что
    его значения не меняются. Можно попробовать убрать
    этот тип из аксиоматизации.
    \item
    Типы \texttt{pointer} и \texttt{alloc\_table}
    становятся типами-символами, т.к. в их определении
    использовался \texttt{block}. Тип \texttt{memory}
    не меняется.
    \item
    \texttt{alloc\_table} был отображением блоков в их
    <<размеры>>. Теперь вместо блоков у нас указатели.
    И каждому указателю надо сопоставить тот же <<размер
    блока>> и смещение. То есть размер блока и смещение
    зависят от \texttt{alloc\_table} и \texttt{pointer}.
    \item
    Вместо размера блока и смещения можно использовать
    два других числа, взаимно однозначно им соответствующих:
    смещение от указателя к началу блока (\texttt{offset\_min})
    и к концу блока (\texttt{offset\_max}).
    \end{itemize}
    \end{frame}

    \begin{frame}[fragile]{Символы}
    Часть бывших не-символов становятся символами и наоборот:
    \begin{lstlisting}
type pointer 't
type alloc_table 't
function offset_min (a: alloc_table 't)
                    (p: pointer 't): int
function offset_max (a: alloc_table 't)
                    (p: pointer 't): int
function shiftP (p: pointer 't)(n: int): pointer 't
function subP(p p2: pointer 't): int
predicate same_block (p p2: pointer 't)
predicate valid (a: alloc_table 't)(p: pointer 't)
  = offset_min a p <= 0 <= offset_max a p
    \end{lstlisting}
    \end{frame}

    \begin{frame}{Остальное}
    Целиком аксиоматизацию можно посмотреть в
    библиотеке инструмента \textsl{AstraVer} примерно тут:
    \texttt{.opam/4.07.1/lib/astraver/why3/core.mlw}
    \end{frame}
\end{document}
