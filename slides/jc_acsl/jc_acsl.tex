\documentclass[hyperref={unicode=true}]{beamer}
\usepackage[utf8]{inputenx}
\usepackage[russian]{babel}

\usepackage{multicol}

\usepackage[pgf]{dot2texi}
\usepackage{tikz}
\usetikzlibrary{shapes, arrows}

\usepackage{listings}
\usepackage{graphicx}

\usepackage{comment}

\title{Лекция 12. Язык ACSL. Особенности
дедуктивной верификации в системе Frama-C.}
\author{}
\date{}

\usetheme{Warsaw}

\AtBeginSection[] {
	\begin{frame}{Содержание}
		\tableofcontents[currentsection]
	\end{frame}
}
%\overfullrule=5pt

\begin{document}
	\begin{frame}{}
		\titlepage
	\end{frame}

    \begin{frame}{Цель лекции}
    Познакомиться с языком формальной спецификации
    программ на языке Си. Увидеть особенности
    формальной спецификации и верификации на уровне
    языка Си с учетом построения модели программы
    в системе Frama-C.
    \end{frame}

    \section{Система Frama-C}

    \begin{frame}{Frama-C}
    \begin{itemize}
    \item Система статического анализа Си-программ.
    \item Состоит из ядра и плагинов. Ядро -- это фронтенд
    анализа, плагин -- бекэнд анализа.
    \item Один из плагинов (AstraVer) -- дедуктивная верификация.
    \item Плагины могут взаимодействовать друг с другом
    для компенсации недостатков друг друга. Пример:
    есть плагин, который сам выводит несложные инварианты цикла,
    и уже на этой основе другой плагин сообщает об ошибке.
    Всё это делается полностью автоматически.
    \item \texttt{frama-c -av file.c ...}
    \end{itemize}
    \end{frame}

    \begin{frame}{Работа на уровне исходного кода}
    \begin{itemize}
    \item Исходный код может быть снабжен аннотациями
    для более точного анализа. Для дедуктивной верификации
    аннотациями записывается спецификация, инварианты цикла
    и т.п.
    \item Причем пользователь Frama-C работает
    исключительно на уровне исходного кода (иначе будет тяжело
    автоматизировать комбинирование плагинов).
    \item На уровне исходного кода не доступна его модель на WhyML.
    Эта модель может иметь существенные особенности по сравнению
    с исходным Си-кодом, важные для верификации.
    \end{itemize}
    \end{frame}

    \begin{frame}{Модель кода WhyML не доступна}
    \begin{itemize}
    \item (-) Например, нельзя написать лемму, в которой
    используются функциональные символы из модели памяти.
    \item (+) Плагин \textsl{AstraVer} может самостоятельно
    строить модель программы, применяя различные
    оптимизации для более эффективной верификации.
    \end{itemize}
    \end{frame}

    \section{Язык ACSL}

    \begin{frame}{Что такое ACSL}
    \begin{itemize}
    \item
    ACSL -- ANSI/ISO C Specification Language.
    \item
    Frama-C использует подмножество языка ACSL.
    \item
    Исходный код дополняется \emph{аннотациями}:
    комментариями \texttt{/*@ ... */}
    \end{itemize}
    \end{frame}

    \begin{frame}{Аннотации ACSL для спецификации функции}
    Спецификация функции -- это аннотация, которая
    расположена перед заголовком функции.
    \begin{itemize}
    \item
    requires expr; -- предусловие (можно несколько requires или ни одного)
    \item
    ensures expr; -- постусловие (можно несколько ensures или ни одного)
    \item
    decreases expr; -- оценочная функция для рекурсивной функции
    (фундированное множество -- то же, что в WhyML)
    \item
    Синтаксис выражений похож на WhyML, но он сделан более
    похожим на Си (синтаксис кванторов!).
    \end{itemize}
    \end{frame}

    \begin{frame}{Типы данных}
    \begin{itemize}
    \item
    \texttt{integer} -- бесконечный целый тип (только для спецификаций)
    \item
    Все Си-шные типы тоже есть
    \item
    В операциях с \texttt{int} и \texttt{integer} происходит преобразование
    \texttt{int} к \texttt{integer}
    \end{itemize}
    \end{frame}

    \begin{frame}{Модель памяти недоступна}
    \begin{itemize}
    \item
    Модель памяти недоступна, поэтому для указателей
    надо использовать Си-шное разыменование, сложение, вычитание,
    сравнение.
    \item
    Конструкции \texttt{$\backslash$old}, \texttt{$\backslash$at}
    \item
    Метки памяти -- это Си-метки + \texttt{Pre}, \texttt{Here}, \texttt{Post}
    \item
    Не путать \texttt{$\backslash$at(*p,L)} и \texttt{*$\backslash$at(p,L)}
    \end{itemize}
    \end{frame}

    \begin{frame}{Спецификация фрейма функции}
    \begin{itemize}
    \item
    Модель памяти недоступна, поэтому для фрейма ввели
    дополнительные части спецификации функции:
    \item
    assigns Locations; -- валидная память за пределами Locations
    не меняет своего значения при выходе из функции
    \item
    allocates Locations; -- валидная память за пределами Locations
    не меняет своего статуса аллоцированности при выходе из функции
    (если была освобождена, остается освобожденной; была выделенной,
     остается выделенной); Locations вычисляется в состоянии памяти
    при возврате из функции
    \item
    frees Locations; -- то же, что allocates, но Locations
    вычисляется при входе в функцию
    \end{itemize}
    \end{frame}

    \begin{frame}{Спецификация циклов}
    \begin{itemize}
    \item
    Аннотация цикла записывается перед циклом. Точки сечения,
    фундированное множество -- те же, что в WhyML.
    \item
    \texttt{loop invariant expr;} -- индуктивное утверждение
    (может быть несколько или отсутствовать)
    \item
    \texttt{loop variant expr;} -- оценочная функция
    (если ее нет, то она равна 0)
    \item
    \texttt{loop assigns Locations;} -- фрейм цикла
    \end{itemize}
    \end{frame}

    \begin{frame}{Символы}
    \begin{itemize}
    \item
    Типы-символы, функциональные и предикатные символы, аксиомы
    пишутся в аннотациях \texttt{axiomatic}
    \item
    Функциональный символ начинается со слова \texttt{logic}
    \item
    Предикатный символ начинается со слова \texttt{predicate}
    \item
    Аксиома начинается со слова \texttt{axiom}
    \item
    Полиморфные типы не поддерживаются
    \end{itemize}
    \end{frame}

    \begin{frame}{Снова без модели памяти}
    \begin{itemize}
    \item
    Модель памяти не доступна даже в axiomatic, это нужно учитывать.
    \item
    Если нужны метки памяти для функциональных и предикатных символов,
    их надо писать в фигурных скобках после имени символа, а потом
    использовать в конструкции \texttt{$\backslash$}at.
    \item
    Меток может быть несколько -- и это удобно! (для разыменования
            каждого указателя своя метка -- и можно использовать
            такой предикатный символ в очень разном контексте,
            не создавая много однотипных предикатных символов)
    \end{itemize}
    \end{frame}

    \begin{frame}{Но есть предикаты, входящие в язык}
    \begin{itemize}
    \item
    \texttt{$\backslash$valid} -- валидность указателя
    \item
    \texttt{$\backslash$offset\_min}, \texttt{$\backslash$offset\_max}
    \item
    \texttt{$\backslash$base\_addr} -- только для выражения, что два
    указателя находятся в одном блоке (совпадают базовые адреса)
    \item
    \texttt{$\backslash$allocable}, \texttt{$\backslash$freeable}
    \end{itemize}
    \end{frame}

    \begin{frame}{Аннотации для верификации}
    \begin{itemize}
    \item
    \texttt{assert expr;}
    \item
    \texttt{ghost}-метки и локальные переменные (тип - только Си-шный),
    \texttt{ghost}-блоки (только с Си-шным кодом): это из-за того, что
    в Frama-C фронтенд кода принимает только Си-код (не ghost-аннотация
    анализируется фронтендом отдельно, поэтому с ней нет этих проблем)
    \item
    \texttt{ghost}-функции
    \item
    \texttt{lemma} -- леммы
    \end{itemize}
    \end{frame}

    \begin{frame}{Behavior}
    \begin{itemize}
    \item
    Можно указать у assert, что он нужен для доказательства
    определенной аннотации ensures
    \item
    То же можно сделать с loop invariant
    \item
    Ensures надо поместить в behavior
    \item
    Синтаксис смотрите в документации по ACSL
    \end{itemize}
    \end{frame}

    \begin{frame}[fragile]{Лемма-функции}
    Пример:

    \begin{lstlisting}
/*@ ghost
     /@ lemma
          requires ....
          ensures ....
     @/
     void lemmafunction(....) {
      ....
     }
*/
    \end{lstlisting}
    \end{frame}

    \begin{frame}{Построение теорий WhyML и их зависимости}
    \begin{itemize}
    \item
    Каждый axiomatic и глобальная лемма становятся
    отдельной теорией.
    \item
    В теорию импортируются все теории, в которых
    объявляются символы, используемые в первой теории.
    \item
    Для более тонкого управления импортированием
    надо объявить символ и воспользоваться им в нужном месте.
    \item
    В условиях верификации используются все леммы,
    расположенные до верифицируемой функции.
    \end{itemize}
    \end{frame}

    \begin{frame}{Триггеры}
    \begin{itemize}
    \item
    В кванторах ACSL нет триггеров.
    \item
    Солвер будет выбирать триггер, исходя из квантора, каким
    он будет в WhyML! Квантор может сильно отличаться
    от его ACSL-варианта.
    \end{itemize}
    \end{frame}

    \begin{frame}{Переносимость Си-программ и верификации}
    \begin{itemize}
    \item
    В Си размеры типов выбирает платформа, а не язык.
    \item
    Frama-C фиксирует размеры типов во фронтенд анализе
    (опциями можно настраивать эти размеры). Дедуктивная
    верификация делается с этими размерами типов.
    \item
    Последовательность вычислений тоже фиксируется
    фронтендом Frama-C.
    \end{itemize}
    \end{frame}
\end{document}
