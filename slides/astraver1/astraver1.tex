\documentclass[hyperref={unicode=true}]{beamer}
\usepackage[utf8]{inputenx}
\usepackage[russian]{babel}

\usepackage{multicol}

\usepackage[pgf]{dot2texi}
\usepackage{tikz}
\usetikzlibrary{shapes, arrows}

\usepackage{listings}
\usepackage{graphicx}

\usepackage{comment}

\title{Лекция 12. Инструмент AstraVer: модель памяти,
статическое разрешение синонимии указателей}
\author{}
\date{}

\usetheme{Warsaw}

\AtBeginSection[] {
	\begin{frame}{Содержание}
		\tableofcontents[currentsection]
	\end{frame}
}
%\overfullrule=5pt

\begin{document}
	\begin{frame}{}
		\titlepage
	\end{frame}

    \begin{frame}{Цель лекции}
    Обсудить некоторые проблемы дедуктивной верификации
    Си-программ и их решения, реализованные
    в инструменте AstraVer
    \end{frame}

    \section{Модель памяти в инструменте AstraVer}

    \begin{frame}{Убран тип block}
    \begin{itemize}
    \item
    Тип \texttt{block} отличается от других тем, что
    его значения не меняются. Можно попробовать убрать
    этот тип из аксиоматизации.
    \item
    Типы \texttt{pointer} и \texttt{alloc\_table}
    становятся типами-символами, т.к. в их определении
    использовался \texttt{block}. Тип \texttt{memory}
    не меняется.
    \item
    \texttt{alloc\_table} был отображением блоков в их
    <<размеры>>. Теперь вместо блоков у нас указатели.
    И каждому указателю надо сопоставить тот же <<размер
    блока>> и смещение. То есть размер блока и смещение
    зависят от \texttt{alloc\_table} и \texttt{pointer}.
    \item
    Вместо размера блока и смещения можно использовать
    два других числа, взаимно однозначно им соответствующих:
    смещение от указателя к началу блока (\texttt{offset\_min})
    и к концу блока (\texttt{offset\_max}).
    \end{itemize}
    \end{frame}

    \begin{frame}[fragile]{Символы}
    Часть бывших не-символов становятся символами и наоборот:
    \begin{lstlisting}
type pointer 't
type alloc_table 't
function offset_min (a: alloc_table 't)
                    (p: pointer 't): int
function offset_max (a: alloc_table 't)
                    (p: pointer 't): int
function shiftP (p: pointer 't)(n: int): pointer 't
function subP(p p2: pointer 't): int
predicate same_block (p p2: pointer 't)
predicate valid (a: alloc_table 't)(p: pointer 't)
  = offset_min a p <= 0 <= offset_max a p
    \end{lstlisting}
    \end{frame}

    \begin{frame}{Остальное}
    Целиком аксиоматизацию можно посмотреть в
    библиотеке инструмента \textsl{AstraVer} примерно тут:
    \texttt{.opam/4.07.1/lib/astraver/why3/core.mlw}
    \end{frame}

    \begin{frame}[fragile]{Изучите сгенерированную модель программы}
    Запустите AstraVer на этом примере
    и изучите модель программы:
    \begin{lstlisting}
//@ requires \valid(v) && \valid(i);
void at(int *v, int n, int *i) {
    if (i - v > 2) {
        *v = *++i;
    }
}    \end{lstlisting}
    \end{frame}

    \section{Component-as-array}

    \begin{frame}{Формулировка проблемы}
    \begin{itemize}
    \item Как моделировать Си-структуры? Массивы Си-структур?
    \item В WhyML есть структуры, но их нет в солверах
    \item Значит, тип структуры будет типом-символом, поля будут
    функциональными символами
    \item Но при изменении внутреннего элемента массива / поля
    получаются очень объемные условия верификации (проверьте!)
    \end{itemize}
    \end{frame}

    \begin{frame}{Component-as-array}
    \begin{itemize}
    \item Заметим, что отдельные поля одной структуры не могут
    находиться в одной области памяти (это не union).
    \item Решение проблемы: отдельный \texttt{memory map} для
    каждого поля структуры.
    \item (+) Упрощение верификации
    \item (-) Усложнение генерации модели программы (нужна
            автоматизация)
    \end{itemize}
    \end{frame}

    \begin{frame}[fragile]{Изучите сгенерированную модель программы}
    Запустите AstraVer на этом примере
    и изучите, как он разделил структуру
    по полям:
    \begin{lstlisting}
typedef struct {
    int capacity, size, *data;
} Vector;
//@ requires \valid(v) && \valid(v->data);
void at(Vector *v, int i) {
    if (v->size < v->capacity &&
        0 <= i < v->size) {
        v->data[i] = 0;
        ++ v->size;
    }
}    \end{lstlisting}
    \end{frame}

    \section{Статическое разрешение синонимии указателей}

    \begin{frame}{Синонимия указателей}
    \begin{itemize}
    \item
    Два указателя являются синонимами (алиасами), если
    изменение памяти по одному указателю влияет на
    память по другому указателю
    \item
    Проблема разрешения синонимии -- определение того, какие
    указатели в тексте программы не являются синонимами.
    \item
    Это позволяет генерировать более эффективный код компилятору
    (запоминать считанные значения и не считывать их повторно)
    \item
    Известная исследовательская проблема в компиляторных
    технологиях
    \end{itemize}
    \end{frame}

    \begin{frame}{Синонимия указателей в дедуктивной верификации}
    \begin{itemize}
    \item
    Важно разрешать синонимию и для дедуктивной верификации
    \item
    Если указатели не являются синонимами, то можно использовать
    для них разные переменные для модели памяти: <<изменение>>
    одной переменной не повлияет на другую переменную
    \item
    То есть это повышает эффективность разрешения условий верификации
    \item
    Постановка задачи: сопоставить каждому разыменованию каждого
    указателя некоторую переменную - модель памяти, чтобы у заведома
    несинонимов были разные модели памяти
    \end{itemize}
    \end{frame}

    \begin{frame}{Регионы}
    \begin{itemize}
    \item
    Разделяем все указатели на классы эквивалентности (\emph{регионы}),
    в одном классе находятся указатели, которые могут совпадать
    \item
    Изменение памяти по указателю из одного региона не
    изменяет память по указателям всех остальных регионов
    \item
    Поэтому для региона делаем отдельные переменные
    \texttt{alloc\_table} и \texttt{memory}
    \end{itemize}
    \end{frame}

    \begin{frame}{Один класс эквивалентности}
    Когда два указателя должны находиться в одном регионе:
    \begin{itemize}
    \item
    когда они сравниваются в коде \emph{даже на неравенство}
    \item
    когда они сравниваются в спецификации
    \item
    когда какой-то указатель из региона одного
    указателя сравнивается с каким-то указателем
    из региона второго указателя
    \item
    когда один получен из другого прибавлением числа,
    \emph{каким бы оно ни было} (иначе алгоритм расстановки
    регионов по указателям не будет статическим)
    \end{itemize}
    \end{frame}

    \begin{frame}[fragile]{Пример}
    Расставьте регионы в следующем примере кода:
    \begin{lstlisting}
    int *a, *q, *r; ...
    int *p = a;
    while (p < q - 1) {
        if (*p == *r) ++*p;
        ++p;
    }
    \end{lstlisting}
    \end{frame}

    \begin{frame}{Регионы для функций}
    \begin{itemize}
    \item
    Нужны дополнительные неявные параметры для региона
    \item
    Для каждого явного параметра-указателя нужен
    параметр-регион, но некоторые параметры-указатели
    могут использовать один и тот же параметр-регион
    \item
    Связь между использованием параметра-указателя
    и использованием параметра-региона делается
    статически
    \item
    Если есть вызов этой функции, где у двух
    параметров-указателей один регион, то них
    будет общий параметр-регион
    \item
    Иначе у параметров-указателей будут разные
    параметры-регионы
    \end{itemize}
    \end{frame}

    \begin{frame}[fragile]{Пример -- отдельные регионы}
    В этом примере у a и b будут свои отдельные параметры-регионы
    \begin{lstlisting}
void f(int *a, int *b)
{
    *a = 0;
    *b = 0;
}
void g()
{
    int m[10];
    int n;
    f(&m[2], &n);
}    \end{lstlisting}
    \end{frame}

    \begin{frame}[fragile]{Пример -- совместный регион}
    В этом примере у a и b будет один параметр-регион
    \begin{lstlisting}
void f(int *a, int *b)
{
    *a = 0;
    *b = 0;
}
void g()
{
    int m[10];
    int n;
    f(&m[2], &m[3]);
}    \end{lstlisting}
    \end{frame}

    \begin{frame}[fragile]{Упражнение}
    Почему не доказывается assert и что надо сделать
    для его доказательства?
    \begin{lstlisting}
/*@ requires \offset_min(s) == 0;
    requires \offset_max(s) == 0;
    ensures \base_addr(s) != \base_addr(\result);
*/
void *f(void *s)
{
    void *u = malloc(10);
    return u;
}
    \end{lstlisting}
    \end{frame}

    \begin{frame}[fragile]{Упражнение}
    Почему возникает ошибка при построении модели тут?
    Исправьте ее (метка памяти -- не то же, что регион!).
    \begin{lstlisting}
/*@
    axiomatic getValue {
        logic int *value(int *ar, integer);
    }

    predicate test1(int *ar, integer i) =
        *value(ar, i) == 0;
    predicate test2(int *ar, integer i) =
        test1(ar, i);
*/
    \end{lstlisting}
    \end{frame}

    \section{Спецификация синонимии}

    \begin{frame}{Решение в ACSL}
    \begin{itemize}
    \item
    В ACSL есть предикат \texttt{\textbackslash{}separated}
    от указателей. Он означает, что эти указатели не указывают
    на одни и те же области памяти (их разыменования
    независимы).
    \item
    Ответьте, почему этот предикат в общем случае не
    выразим в модели памяти AstraVer
    \end{itemize}
    \end{frame}

    \begin{frame}{Решение в AstraVer}
    \begin{itemize}
    \item
    \texttt{\textbackslash{}base\_addr(p1) != \textbackslash{}base\_addr(p2)
    } -- указатели не принадлежат одному блоку
    \item
    \texttt{p1 == p2 || p1 != p2} -- разместить указатели
    в одном регионе
    \item
    Разделение по регионам вместо предиката \texttt{\textbackslash{}separated}
    \end{itemize}
    \end{frame}

    \begin{frame}{Модульная верификация и разделение по регионам}
    \begin{itemize}
    \item
    Вычисление регионов (т.е. переменные для модели памяти) требуют
    наличия всего исходного кода. А если есть только заголовочный файл?
    \item
    Для каждого файла-реализации для заголовочного файла
    придется верификацию всей программы делать сначала!
    \item
    AstraVer умеет делать некоторые предположения о регионах,
    исходя из заголовка (например, (не)константность указателя)
    \end{itemize}
    \end{frame}

\end{document}
