\documentclass[hyperref={unicode=true}]{beamer}
\usepackage[utf8]{inputenx}
\usepackage[russian]{babel}

\usepackage{multicol}

\usepackage[pgf]{dot2texi}
\usepackage{tikz}
\usetikzlibrary{shapes, arrows}

\usepackage{listings}
\usepackage{graphicx}

\usepackage{comment}

\title{Лекция 14. Инструмент AstraVer: приведение
типов указателей, noembed, итоги}
\author{}
\date{}

\usetheme{Warsaw}

\AtBeginSection[] {
	\begin{frame}{Содержание}
		\tableofcontents[currentsection]
	\end{frame}
}
%\overfullrule=5pt

\begin{document}
	\begin{frame}{}
		\titlepage
	\end{frame}

    \begin{frame}{Цель лекции}
    Обсудить некоторые проблемы построения модели
    Си-программ и их решения, реализованные
    в инструменте AstraVer
    \end{frame}

    \begin{frame}[fragile]{Типы для примеров}
    Далее будут использоваться следующие типы (из
    практического задания):
    \begin{lstlisting}{C}
typedef struct {
    int a, b;
} Key;
typedef struct {
    int c, d;
} Value;
typedef struct {
    Key key;
    Value value;
} Item;
    \end{lstlisting}
    \end{frame}

    \section{Приведение типов указателей}

    \begin{frame}{Иерархии типов указателей в Си}
    \begin{itemize}
    \item \texttt{void *} -- чем не аналог типа-предка
    для любого типа-указателя?
    \item еще примеры <<наследования>> типов в Си?
    (первым полем структуры является другая структура)
    \end{itemize}
    \end{frame}

    \begin{frame}{Модель Си-программы с наследованием}
    \begin{itemize}
    \item
    Когда надо применять правила моделирования типа
    как потомка?
    Вспомните эти правила из предыдущей лекции.
    \item
    Решение AstraVer: только если в исходном коде или
    спецификации встречается операция приведения
    \item
    NB! Приведение между \texttt{void *} и остальными
    типами указателей может быть неявным!
    \item
    Если в условии верификации нет ничего про subtag
    нужных типов или генерируются условия верификации
    про приведение к \texttt{char *}, значит, возможно,
    в коде не хватает выражения с нужным приведением типа!
    \end{itemize}
    \end{frame}

    \begin{frame}[fragile]{Еще пример неявного приведения типа}
    Изучите модель программы, которую сгенерирует AstraVer:
    \begin{lstlisting}
//@ ensures \result == 0;
int main(void) {
    int m[10];
    int n;
    //@ assert &n != &m[0];
    return 0;
}
    \end{lstlisting}
    \end{frame}

    \begin{frame}{Операция взятия адреса и массивы}
    \begin{itemize}
    \item
    Определение массива превращается в malloc и в конце блока кода,
    где объявлена переменная, в free
    \item
    Переменная, от которой берется адрес, превращается в массив
    размера 1 (а он в malloc)
    \item
    Если адрес от переменной не берется, то она не превращается
    в массив (в этом превращении нет необходимости)
    \end{itemize}
    \end{frame}

    \begin{frame}{Корректность приведения типов указателей}
    \begin{itemize}
    \item когда приведение \texttt{void *} к
    типу \texttt{T *} корректно? (последующее разыменование
    полученного указателя будет корректно)
    \item когда это приведение и не потребует переинтерпретации
    памяти?
    \end{itemize}
    \end{frame}

    \begin{frame}{Аналог instanceof в AstraVer}
    \begin{itemize}
    \item
    Чтобы доказать корректность приведения \texttt{void *p}
    к некоторому типу \texttt{T *}, надо доказать, что
    \texttt{p} на самом деле указывает на \texttt{T} (или
    его потока) -- в Java это называется instanceof
    \item
    AstraVer поддерживает такой синтаксис в спецификациях,
    означающий instanceof:
    \texttt{\textbackslash{}typeof(p) <: \textbackslash{}type(T)}
    \end{itemize}
    \end{frame}

    \begin{frame}{\texttt{void *} в стандартной библиотеке}
    \begin{itemize}
    \item
    \texttt{void *} часто используется в стандартной библиотеке
    языка Си (как тип формального параметра функции, чтобы
    ее можно было применять к различным типам указателей)
    \item
    Для некоторых функций стандартной библиотеки
    AstraVer содержит шаблонные спецификации: надо сказать,
    какой динамический тип указателя, тогда будет сгенерирована
    спецификация с этим типом указателя
    \end{itemize}
    \end{frame}

    \section{Вложенные типы}

    \begin{frame}{Component-as-array: тип поля - скалярный}
    \begin{itemize}
    \item
    Вспомним идею о том, что изменение одного поля структуры
    не влияет на остальные поля структуры. Иначе говоря,
    указатели на разные поля одной структуры не являются синонимами.
    Это используется для разделения модели памяти на независимые
    части.
    \item
    Для типа \texttt{Value} будут построены такие типы и переменные:
        \begin{itemize}
        \item \texttt{type Value}
        \item \texttt{constant cFieldMemory: memory (pointer Value) int}
        \item \texttt{constant dFieldMemory: memory (pointer Value) int}
        \end{itemize}
    \item
    Зачем тут pointer? (две отдельные переменные типа \texttt{Value *},
    изменяем значение по одному указателю, другой указатель не трогали,
    но значение по второму указателю тоже должно измениться - как это
    промоделировать без pointer?)
    \end{itemize}
    \end{frame}

    \begin{frame}{Component-as-array: нескалярный тип поля}
    \begin{itemize}
    \item
    Теперь строим переменные и типы для \texttt{Item}. Вот так правильно?
        \begin{itemize}
        \item \texttt{type Item}
        \item \texttt{type Key}
        \item \texttt{type Value}
        \item \texttt{constant keyFieldMemory: memory (pointer Item) Key}
        \item \texttt{constant valueFieldMemory: memory (pointer Item) Value}
        \end{itemize}
    \item
    Если да, то как промоделировать \texttt{item->value.c}? Вот так
    правильно?
    \texttt{(select cFieldMemory (select valueFieldMemory item))}
    Откуда взять там \texttt{pointer}?
    \end{itemize}
    \end{frame}

    \begin{frame}{Выводы}
    \begin{itemize}
    \item
    Чтобы повысить эффективность солверов, при моделировании программы
    можно применить подход \texttt{component-as-array}.
    \item
    Тем не менее, синонимы могут быть (два указателя на одно и то же поле).
    Поэтому при моделировании частей нескалярных типов не получается
    выразить их вложение в объемлющий тип. Каждое нескалярное поле структуры
    моделируется указателем на отдельную область памяти.
    \item
    Например, следующее утверждение (верное по стандарту языка Си)
    не докажется: \texttt{assert (void *)\&key->a < (void *)\&key->b;}
    \end{itemize}
    \end{frame}

    \begin{frame}[fragile]{Иногда это может создать трудности}
    Почему assert не доказывается?
    \begin{lstlisting}{C}
//@ requires \valid(items + (0 .. 1));
void prog(Item *items)
{
    items[0].value.c = 0;
    items[1].value.d = 1;
    //@ assert items[0].value.c == 0;
}
    \end{lstlisting}
    \end{frame}

    \begin{frame}{container\_of}
    \begin{itemize}
    \item
    assert не доказывается, потому что нет предусловия, что
    \texttt{\&items[0].value} и \texttt{\&items[1].value}
    не являются синонимами (они находятся в одном регионе,
    т.к. в одном регионе находятся \texttt{\&items[0]} и
    \texttt{\&items[1]}).
    \item
    У каждого item есть свой value. Это можно записать?
    \item
    \texttt{axiom value\_in\_item: \textbackslash{}forall Item *item1, *item2;
    item1 != item2 ==> \&item1->value != \&item2->value;}
    \end{itemize}
    \end{frame}

    \section{Особая трансляция обращения к первому полю}

    \begin{frame}{Обращение к не первому полю}
    \begin{itemize}
    \item
    Пусть \texttt{item} имеет тип \texttt{Item *}.
    \item
    Тогда чтение \texttt{item->value.c} будет моделироваться
    так: \texttt{(select cFieldMemory (select valueFieldMemory item))}.
    \item
    Типы \texttt{Item} и \texttt{Value} не находятся
    в отношении наследования, переменная \texttt{item}
    будет иметь тип \texttt{pointer Item}, переменная
    \texttt{valueFieldMemory} будет иметь тип
    \texttt{memory (pointer Item) (pointer Value)},
    переменная \texttt{cFieldMemory} будет иметь тип
    \texttt{memory (pointer Value) int}.
    \end{itemize}
    \end{frame}

    \begin{frame}{Обращение к первому полю}
    \begin{itemize}
    \item
    Чтение \texttt{item->key.a} можно моделировать так же
    (через два select и два memory), но можно учесть
    наследование и использовать более простое выражение:
    \texttt{(select aFieldMemory item)}. Это совпадает
    с чтением \texttt{((Key *)item)->a}.
    \item
    Иными словами,
    выражение \texttt{\& item->key} можно моделировать так же,
    как и \texttt{(Key *)item}, а (из-за наследования)
    последнее выражение можно моделировать просто как \texttt{item}.
    \item
    Переменная \texttt{item} теперь будет иметь тип
    \texttt{pointer Key} (а не \texttt{pointer Item},
    т.к. здесь нужно иметь возможность привести
    указатель на потомка к указателю на предка). Переменная
    \texttt{aFieldMemory} будет иметь тип
    \texttt{memory (pointer Key) int}.
    \end{itemize}
    \end{frame}

    \begin{frame}{Хитрость с адресной арифметикой}
    \begin{itemize}
    \item
    Пусть \texttt{items} имеет тип \texttt{Item [2]}.
    \item
    Какова модель для \texttt{items[1].key.a}?
    Какова модель для \texttt{(\& items->key)[1].a}?
    \item
    Они совпадают! \texttt{(items + 1)->a}
    Но оба ли исходных выражения корректны? Только первое
    (почему?).
    Как научить модель запрещать некорректное выражение?
    \end{itemize}
    \end{frame}

    \begin{frame}{Модели всё же немного отличаются}
    \begin{itemize}
    \item
    Можно увидеть отличие моделей, если рассмотреть их
    вот так:
        \begin{itemize}
        \item первое выражение: \texttt{((Key *)(items + 1))->a}
        \item второе выражение: \texttt{((Key *) items + 1 )-> a}
        \end{itemize}
    \item
    Чтобы запретить второе, можно добавить еще одно предусловие
    для операции обращения к полю от указателя, полученного
    сдвига другого указателя: что тип блока этих указателей
    должен быть ровно таким, где определено поле а.
    \end{itemize}
    \end{frame}

    \begin{frame}{Как это сделано в AstraVer}
    \begin{itemize}
    \item
    Чтобы увидеть, как эта идея реализована в аксиоматике памяти
    AstraVer, надо открыть файл \texttt{core.mlw}. Найти там
    модуль \texttt{Acc\_offset\_safe} и смотреть val-примитив
    \texttt{acc\_offset}.
    \end{itemize}
    \end{frame}

    \begin{frame}[fragile]{Но идея работает не всегда}
    Идея оказывается слишком жесткой, т.к. в таком коде возникает
    обращение к полю после сдвига и обращение безопасно, но оно
    тоже будет воспринято как некорректное (не докажется одно из
    условий верификации):
    \begin{lstlisting}{C}
//@ requires \valid(items + (0 .. 1));
int prog(Item *items) {
    int offs = 1;
    return (& items->key + 1 - offs)->a;
}
    \end{lstlisting}
    \end{frame}

    \begin{frame}[fragile]{Наконец, noembed}
    \begin{itemize}
    \item
    Поэтому моделирование первого поля как предка некорректно.
    \item
    Чтобы это отменить, надо указать у первого поля атрибут
    noembed (тогда модель станет несколько сложнее, зато
    корректнее):
    \end{itemize}

    \begin{lstlisting}{C}
typedef struct {
    Key key __attribute__((noembed));
    Value value;
} Item;
    \end{lstlisting}
    \end{frame}

    \section{Итоги}

    \begin{frame}{AstraVer на практике}
    \begin{itemize}
    \item
    Цель проекта ИСП РАН -- применять дедуктивную
    верификацию для исходного кода отдельных модулей
    ядра Linux.
    \item
    При этом оказалось, что:
        \begin{enumerate}
        \item
        ядро написано под определенный набор расширений
        языка Си (gcc)
        \item
        ядро использует переинтерпретацию памяти
        \end{enumerate}
    \item
    За основу AstraVer был взят инструмент Jessie. AstraVer
    исправил часть модели памяти Jessie, добавил некоторые
    новые возможности (
    \url{http://astraver.linuxtesting.org/manuals/features/}).
    Но суть модели памяти не поменялась. Ограничения модели
    остались.
    \item
    Плюс проблема с тем, что модель памяти недоступна на
    уровне ACSL.
    \end{itemize}
    \end{frame}

    \begin{frame}{Новый инструмент}
    \begin{itemize}
    \item
    Поэтому сейчас разрабатывается новый инструмент
    дедуктивной верификации.
    \item
    Он основан на интерактивном прувере Isabelle/HOL,
    чтобы можно было управлять процессом доказательства.
    Причем для спецификации требований и для доказательства
    условий верификации можно было бы использовать
    один и тот же язык.
    \item
    И не менее важное, что в новом инструменте будет
    поддерживаться настоящая модель памяти языка Си.
    \end{itemize}
    \end{frame}

    \begin{frame}{Другие проекты по дедуктивной верификации}
    \begin{itemize}
    \item
    Есть игрушечные проекты, их много, но мы их не рассматриваем.
    Обычно они требуют особой подготовки пользователя и не все
    готовы тратить время на получение этой подготовки.
    \item
    \texttt{VCC} -- проект Microsoft по дедуктивной верификации
    исходного кода гипервизора Hyper-V (исходники на Си).
    Архитектура инструмента та же, что у AstraVer, но
    используется другая модель памяти.
    \item
    \texttt{SPARK} -- дедуктивная верификация Ada. Используется
    много где (например, при создании авионики, например, А350)
    \end{itemize}
    \end{frame}

    \begin{frame}{Формальные методы на практике}
    \begin{itemize}
    \item
    Основная проблема формальных методов -- это большой объем
    спецификаций (их надо написать), большие трудозатраты
    на формальную верификацию, специфические навыки по
    спецификации и по верификации.
    \item
    Опыт показывает, что одно только написание формальных
    спецификаций дает очень много:
        \begin{itemize}
        \item
        это позволяет систематизировать требования к
        создаваемой программе
        \item
        это требует продумывания поведения во всех
        ситуациях, даже в тех, о которых раньше не
        задумывались
        \end{itemize}
    \item
    Иногда выписывание формальной спецификации приводит
    к пониманию того, что в дизайне системы есть ошибка,
    и эта ошибка исправляется до того, как система
    уже реализована!
    \end{itemize}
    \end{frame}

    \begin{frame}{Формальные методы на практике}
    \begin{itemize}
    \item
    Еще одно направление в практике -- это комбинация
    формальных спецификаций и легковесной верификации
    (формального тестирования). Формальная спецификация
    позволяет не пропустить важные тестовые случаи.
    \item
    Еще одно направление в практике -- это
    многоуровневая формальная спецификация и верификация.
    То есть пишется сначала очень абстрактная спецификация,
    но описывающая свой уровень подробно. Затем
    абстрактные понятия уточняются и доказывается корректность
    уточнения.
    \end{itemize}
    \end{frame}

    \begin{frame}{Формальные методы в России}
    \begin{itemize}
    \item
    Многоуровневые спецификации используются в следующих
    организациях:
        \begin{itemize}
        \item РусБИТех -- спецификация отдельных модулей
        безопасности в операционной системе AstraLinux,
        доказательство соответствия уровней (Event-B, Pro-B),
        тестирование
        реализаций и дедуктивная верификация (AstraVer)
        \item МЦСТ -- спецификация отдельных модулей
        безопасности в операционной системе Эльбрус
        \end{itemize}
    \item
    По заказу ГосНИИАС разрабатывается ОСРВ с открытым
    кодом JetOS, удовлетворяющая стандарту ARINC653. ОС
    проектируется так, чтобы как можно быстрее пройти
    сертификацию.
    \end{itemize}
    \end{frame}

\end{document}
