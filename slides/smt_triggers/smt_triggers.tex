\documentclass[hyperref={unicode=true}]{beamer}
\usepackage[utf8]{inputenx}
\usepackage[russian]{babel}

\usepackage{multicol}

\usepackage[pgf]{dot2texi}
\usepackage{tikz}
\usetikzlibrary{shapes, arrows}

\usepackage{listings}
\usepackage{graphicx}

\usepackage{comment}

\title{Лекция 5. Триггеры}
\author{}
\date{}

\usetheme{Warsaw}

\AtBeginSection[] {
	\begin{frame}{Содержание}
		\tableofcontents[currentsection]
	\end{frame}
}
%\overfullrule=5pt

\begin{document}
	\begin{frame}{}
		\titlepage
	\end{frame}

    \begin{frame}{Цель лекции}
    Научиться эффективно использовать SMT-солверы
    для дедуктивной верификации.
    \end{frame}

    \section{Мотивация}

	\begin{frame}[fragile]{Пример для дедуктивной верификации}
    \vspace*{-4mm}
    \begin{multicols}{2}

	\huge
	\begin{dot2tex}[options=-traw]
	digraph G{
		d2tgraphstyle="scale=0.3, transform shape";

		/* nodes by levels */
		node[shape=rectangle, height=1];
		START[style=rounded, width=4, texlbl="$\begin{matrix}START:\\(y_1,~y_2) \leftarrow (x,~0)\end{matrix}$"];
        CALL1[width=4, texlbl="$\begin{matrix}CALL:\\y_2 \leftarrow DIV(x + 1,~2)\end{matrix}$"];
		JOIN[shape=point, width=0, label=""];
        COND[style=rounded, width=2, texlbl="$y_2 \geq y_1$"];
        CALL2[width=4, texlbl="$\begin{matrix}CALL:\\y_1 \leftarrow DIV(x,~y_2)\end{matrix}$"];
        CALL3[width=4, texlbl="$\begin{matrix}CALL:\\y_1 \leftarrow DIV(y_1 + y_2,~2)\end{matrix}$"];
		ASSIGN[width=3, texlbl="$(y_1,~y_2) \leftarrow (y_2,~y_1)$"];
        HALT[style=rounded, width=3, texlbl="$\begin{matrix}HALT:\\  z \leftarrow y_1\end{matrix}$"];

		/* edges */
		node [shape=point, width=0, label=""];
        START -> CALL1;
        CALL1 -> JOIN [arrowhead=none];
        JOIN -> COND [weight=8];
		{ rank=same; p1 -> JOIN; }
		p1 -> p5 [weight=8, arrowhead=none];
		{ rank=same; p3 -> COND [label="F", arrowhead=none]; COND -> p4 [label="T", arrowhead=none]; }
		p3 -> CALL2 -> CALL3 -> ASSIGN [weight=8];
		p4 -> HALT [weight=8];
		{ rank=same; ASSIGN ; HALT; }
		ASSIGN -> p6 [weight=8, arrowhead=none];
	    { rank=same; p5 -> p6 [arrowhead=none]; }
	}
	\end{dot2tex}

	\normalsize

    $D_{x} = D_{y_1} = D_{y_2} = D_{z} = \mathbb{Z}$.
    Надо доказать полную корректность этой блок-схемы
    относительно спецификации $\varphi(x) = x \geq 0$,
    $\psi(x,~z) = (z^2 \leq x < (z + 1)^2)$, если
    сигнатуре DIV сопоставлена спецификация
    $\varphi_D(a_1,~a_2) = (a_1 \geq 0 \land a_2 > 0)$,
    $\psi_D(a_1,~a_2,~r) = (0 \leq a_1 - r * a_2 < a_2)$.
    \end{multicols}
	\end{frame}

    \begin{frame}{Ход решения}
    \begin{itemize}
    \item
    Докажите полную корректность этой блок-схемы
    <<на листочке>> самостоятельно.
    \item
    Выпишите условия верификации на языке Why3.
    \item
    Используйте Why3IDE для доказательства условий
    верификации при помощи солверов \textsl{Alt-Ergo},
    \textsl{CVC4}, \textsl{Z3}.
    \end{itemize}
    \end{frame}

    \begin{frame}{Первые выводы}
    \begin{itemize}
    \item
    Вы видите, что солверы не могут доказать некоторые
    из истинных условий верификации.
    \item
    Индуктивное утверждение можно записать с разными
    кванторами и вы видите, что с одними кванторами солвер
    доказывает больше условий верификации, чем с другими.
    \item
    Во всем этом надо разобраться.
    \end{itemize}
    \end{frame}

    \section{Подробнее о солверах}

    \begin{frame}{Солверы}
    \begin{itemize}
    \item
    Для <<доказательства условий верификации>> мы используем \emph{солверы}.
    \item
    Солвер -- это алгоритм решения <<уравнений>> определенного вида
    (типовое <<уравнение>> -- это формула со свободными переменными)
    или программа, реализующая этот алгоритм.
    \item
    Примеры: солвер дифференциальных уравнений, SAT-солвер,
    солвер тригонометрических уравнений, солвер логических уравнений.
    \item
    Солвер ожидает формулу на известном ему языке, т.е. с определенным
    набором операций. На них рассчитан алгоритм солвера.
    Их еще называют <<теорией>> (тригонометрия, логика и т.п.).
    \end{itemize}
    \end{frame}

    \begin{frame}{Солверы (2)}
    \begin{itemize}
    \item
    У солвера есть 3 варианта ответа:
    \begin{itemize}
    \item решение найдено, и оно предъявляется
    \item солвер определил, что решений нет, формула противоречива
    \item солвер не знает, какой из первых двух пунктов имеет место быть
    (не для всех уравнений ведь есть метод решения)
    \end{itemize}
    \end{itemize}
    \end{frame}

    \begin{frame}{SMT-солвер}
    \begin{itemize}
    \item
    SMT-солвер -- это комбинация солверов. То есть это программа для
    решения <<уравнений>> из комбинации теорий.
    \item
    Как скомбинировать солверы?
    Например, получив формулу, можно сделать <<замену переменных>>, как
    вы это делали при решении уравнений на вступительных экзаменах, и
    получить несколько задач, каждая из своей теории.
    \item
    Мы уже использовали теорию целых чисел.
    \end{itemize}
    \end{frame}

    \begin{frame}{Солвер как прувер}
    Как доказать истинность формулы (условия верификации) при помощи солвера?
    \begin{itemize}
    \item
    Условие верификации -- это формула с квантором всеобщности (как минимум
    по значениям входных переменных).
    \item
    Надо взять ее отрицание
    и применить сколемизацию. Тогда получится формула со свободными
    переменными.
    \item
    Если солвер определит, что у этой формулы нет решений,
    то можно считать доказанным истинность исходной формулы.
    \item
    Если решение есть, то исходная формула противоречива и есть
    контрпример.
    \item
    Солвер может ответить, что не знает, есть ли решение.
    \end{itemize}
    \end{frame}

    \begin{frame}{Пруверы}
    \begin{itemize}
    \item
    Альтернатива солверам -- пруверы. Они проверяют данный пользователем
    вывод формулы в заданной логике. Вывод -- это цепочка преобразований
    множества формул из аксиом. Допускаются только определенные
    преобразования (<<правила вывода>>).
    \item
    Proof assistant -- командная среда, помогающая пользователю
    составить вывод. Возможности варьируются: от полной автоматизации
    до лишь записи пошагового вывода, который полностью пишется пользователем.
    \item
    Главное: солверы -- обычно более автоматические, чем пруверы. Поэтому
    мы используем солверы, а не пруверы.
    \item
    Примеры пруверов: \textsl{Coq}, \textsl{HOL/Isabelle}, \textsl{PVS}.
    \end{itemize}
    \end{frame}

    \section{Кванторы и солверы}

    \begin{frame}{Кванторы в солверах}
    \begin{itemize}
    \item
    Пусть есть формула для условия верификации,
    истинность которой надо доказать. Берем отрицание формулы и
    сколемизируем ее. Получаем конъюнкцию множества подформул.
    \item
    Могут остаться кванторы всеобщности в конъюнктах. Что с ними делать?
    Подбирать значения подкванторных переменных и инстанцировать кванторы,
    чтобы получить противоречащие подформулы.
    \end{itemize}
    \end{frame}

    \begin{frame}{Пример с инстанцированием}
    \begin{itemize}
    \item При помощи инстанцирования кванторов докажите противоречивость
    данного набора формул. Все свободные переменные принадлежат $\mathbb{Z}$.
    $\begin{cases}
    x > 0\\
    \forall t \in \mathbb{Z} \cdot t > 0 \Rightarrow x * t = x\\
    \end{cases}$
    \item
    Инстанцируем квантор значением $t = 2$. Получаем новую формулу
    $2 > 0 \Rightarrow x * 2 = x$. После упрощения: $x * 2 = x$.
    Целочисленный солвер это упростит до $x = 0$. Получили
    противоречие с $x > 0$.
    \end{itemize}
    \end{frame}

    \begin{frame}{Выводы из примера}
    \begin{itemize}
    \item
    Как автоматически определить, что можно подставить 2 в качестве $t$?
    Множество значений переменной $t$ бесконечно, его не перебрать.
    \item
    В общем случае, задача выбора подстановок, чтобы
    получить противоречащие формулы, алгоритмически неразрешима...
    \item
    Может быть вместо инстанцирования надо ввести дополнительные утверждения,
    доказать их и использовать? Так делает человек.
    Но солвер не умеет <<придумывать>> дополнительные утверждения.
    \end{itemize}
    \end{frame}

    \begin{frame}{Проблема инстанцирования кванторов всеобщности}
    \begin{itemize}
    \item
    Если солвер не может придумывать, то он должен использовать то, что
    уже имеет, т.е. части самой формулы и части других инстансов кванторов.
    \item
    Бесконтрольное инстанцирование приводит к большому числу ненужных
    формул, причем рост может быть экспоненциальным.
    \item
    Нужна методика ограничивания инстанцирования.
    \end{itemize}
    \end{frame}

    \begin{frame}{Триггеры}
    \begin{itemize}
    \item
    Триггер -- это терм, в котором встречаются
    все подкванторные переменные как свободные термы триггера.
    \item
    Если солвер в процессе работы имеет терм, унифицирующийся с триггером,
    он определяет подстановку для подкванторных переменных квантора и
    инстанцирует квантор.
    \end{itemize}
    \end{frame}

    \begin{frame}{Варианты триггеров}
    \begin{itemize}
    \item
    Мультитриггер -- это множество термов, в совокупности дающих подстановку
    всех подкванторных переменных.
    \item
    Может быть несколько триггеров у одного квантора. Если можно провести
    унификацию хотя бы с одним из них, она производится.
    \end{itemize}
    \end{frame}

    \begin{frame}{Пример}
    \begin{itemize}
    \item
    Пример квантора с триггером (триггер записан в
            квадратных скобках после подкванторных переменных):
        \texttt{forall x y [f x y]. (g x) -> (f (h y) x)}
    \item
    Терм \texttt{f 0 1} унифицируется с триггером и дает такой
    инстанс квантора: \texttt{(g 0) -> (f (h 1) 0)}.
    \item
    В этом инстансе есть терм, который можно унифицировать с триггером.
    Это терм \texttt{f (h 1) 0}. Получаем новый инстанс квантора:
    \texttt{(g (h 1)) -> (f (h 0) (h 1))}.
    \item
    Опять можно инстанцировать аксиому...
    \end{itemize}
    \end{frame}

    \begin{frame}{Бесконечный цикл инстанцирования}
    \begin{itemize}
    \item
    В этом примере можно попасть в бесконечный цикл инстанцирования квантора.
    Это еще одна причина, почему солвер может давать ответ <<не знаю>>.
    \item
    Чтобы не попадать в этот цикл, надо выбрать другой триггер
    или вместо исходного квантора записать другой, эквивалентный ему. Например,
    триггер \texttt{f (h y) x}. Тогда при инстанцировании этого квантора
    не появится новых термов, унифицируемых с триггером.
    \item
    Другая проблема -- слишком требовательный триггер. Солвер может никогда
    не получить нужный терм для унификации с триггером.
    \end{itemize}
    \end{frame}

    \begin{frame}{Генерация триггеров}
    \begin{itemize}
    \item
    Если триггер не указан, солвер обычно выбирает его на основе
    некоторой эвристики.
    \item
    Пример эвристики выбора триггера:
        \begin{enumerate}
        \item это терм, являющийся частью самого квантора;
        \item в терме встречаются все подкванторные переменные;
        \item в терм входит хотя бы один пользовательский символ;
        \item это минимальный терм (в нем самом нельзя выделить
                терм, в который входят все подкванторные формулы).
        \end{enumerate}
    \item
    Такую эвристику используют солверы \textsl{Alt-Ergo}, \textsl{CVC3}.
    \end{itemize}
    \end{frame}

    \begin{frame}{Пример}
    \begin{itemize}
    \item Например, для квантора \texttt{forall x y. (x > y) -> (f (h y) x)}
    таким термом является \texttt{(f (h y) x)}.
    \item Терм \texttt{forall x y. (x > y) -> (f (h y) x)} не является
    минимальным.
    \item В терм \texttt{x > y} не входит пользовательский символ.
    \item В терм \texttt{(h y)} не входят все подкванторные переменные.
    \end{itemize}
    \end{frame}

    \begin{frame}{Отладка автоматического выбора триггера}
    \begin{itemize}
    \item
    Солвер \textsl{Alt-Ergo} позволяет понять, какой триггер он выбрал
    для квантора и как он его использовал.
    \item
    \texttt{alt-ergo -dtriggers input.why} распечатывает
    автоматически сгенерированные триггеры
    \item
    \texttt{alt-ergo -dmatching=1 input.why} распечатывает
    кванторы, которые были инстанцированы (вместо 1 можно написать 2 для
            более подробного вывода)
    \item
    Ввод для \textsl{Alt-Ergo} можно получить из Why3IDE: кнопка Edit
    по попытке доказательства этим солвером и текст будет создан в файле
    в поддиректории с вспомогательными файлами.
    \end{itemize}
    \end{frame}

    \begin{frame}{Отладка выбора триггера в CVC4}
    \begin{itemize}
    \item
    \texttt{cvc4 -d trigger smt2file} -- распечатывает
    сгенерированные триггеры
    \item
    Требуется отладочная версия CVC4 (при конфигурировании
    надо указать режим debug)
    \item
    Файл smt2 можно получить при помощи кнопки Edit
    в Why3IDE.
    \end{itemize}
    \end{frame}

    \begin{frame}{Пример с подстановкой $t = 2$}
    \begin{itemize}
    \item
    Ранее в этой лекции был пример системы формул, где противоречие
    находилось подстановкой $t = 2$. Что будет делать солвер с такой системой?
    \item
    Он выделит триггер $x * t$. Но термов для унификации с триггером нет.
    Значит солвер не сможет инстанцировать квантор. \textsl{Alt-Ergo}
    добавляет в систему еще один квантор (одно утверждение про умножение
    целых чисел) и он только и делает, что инстанцирует этот квантор. Его
    ответ -- <<не знаю>>.
    \item
    Правда, это не означает, что не может быть других эвристик: солверы
    \textsl{CVC4} и \textsl{Z3} находят противоречие.
    \end{itemize}
    \end{frame}

    \begin{frame}{Как доказать?}
    \begin{itemize}
    \item
    Попробуем добиться, чтобы \textsl{Alt-Ergo} тоже находил противоречие.
    Надо добавить истинную формулу в систему формул, которая будет давать
    терм \texttt{x * 2}. Например, \texttt{x * 2 = x + x}.
    \item
    Теперь \textsl{Alt-Ergo} мгновенно находит противоречие.
    \end{itemize}
    \end{frame}

    \begin{frame}{Выводы}
    \begin{itemize}
    \item
    Солверы могут давать ответ <<не знаю>> или зацикливаться, если
    им дают формулы с кванторами всеобщности.
    \item
    Если солвер зациклился или <<не знает>> ответ, посмотрите,
    надо ли ему работать с кванторами всеобщности. Какие триггеры выбрал
    солвер для кванторов всеобщности. Какие он должен сделать подстановки
    для подкванторных переменных. Достаточно ли солверу термов, чтобы
    сделать эти подстановки.
    \item
    Теперь вы можете ответить, почему в примере с дедуктивной верификацией
    квадратного корня солверы доказывают больше или меньше
    условий верификации в зависимости от того, как записано
    индуктивное утверждение.
    \end{itemize}
    \end{frame}

\end{document}

